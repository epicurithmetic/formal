% Load required themes and packages.
\documentclass{beamer}
\usetheme{Pittsburgh}
\usecolortheme{default}
\useinnertheme{default}
\useoutertheme{default}
\usefonttheme{structurebold}
% Math related packges.
\usepackage{amsmath}
\usepackage{cancel}

% Natural Deduction package
\usepackage{proof}

% Package for hyperlinks
\usepackage{hyperref}

% Remove navigation bar
\beamertemplatenavigationsymbolsempty

% Bibliography
% \usepackage[style=verbose]{biblatex}
% \bibliography{deltaBib.bib}
% Command line sequence:
% pdflatex *filename*
% biber *filename*
% pdflatex *filename*

\title{Lecture 8: Revision of PL}
\author{MATH230}
\institute{Te Kura P\=angarau $\vert$ School of Mathematics and Statistics \\ Te Whare W\=ananga o Waitaha $\vert$ University of Canterbury}
\date{}

% Document body starts here.
\begin{document}


% Title frame
\begin{frame}

  \titlepage

\end{frame}

% Table of contents page
\begin{frame}
  \frametitle{Outline}

  \tableofcontents

\end{frame}

\section{Test}

\begin{frame}
	\frametitle{Test Details}
	
	Test date: 17th August at 6:30pm in Ernest Rutherford 465.
	
	\vspace{0.5cm}
	
	Duration: 90 minutes 
	
	\vspace{0.5cm} 
	
	The test will cover the propositional logic topic. Completion and understanding of tutorials 2,3, and 4 is the best preparation for the test. 
	
	\vspace{0.5cm}
	
	Today we will do some revision for the test. 
	
\end{frame}


\section{Examples}

\begin{frame}
	\frametitle{Translate NL to PL}
	
	Translate the following argument into propositional logic and determine whether or not it is valid. 
	
	\vspace{0.5cm}
	
	Sherlock is either at home or on a case. Sherlock is at home. Therefore, he is not on a case. 
	
	\vspace{6cm}
	
	
\end{frame}

\begin{frame}
	\frametitle{Translate NL to PL}
	
	Translate the following argument into propositional logic and determine whether or not it is valid. 
	
	\vspace{0.5cm}
	
	If Watson is home, then Gladstone is home. Therefore, if Watson is not home, then Gladstone is not home.  
	
	\vspace{6cm}
	
\end{frame}

\section{Summary}

\begin{frame}
  \frametitle{What is a proof?}

	{\bf Aim:} Recall that one of our original aims was to be precise about what we mean by a good argument i.e. a good proof. 
	
	\vspace{0.5cm} 
	
	We have come up with two different perspectives: 
	
	\begin{enumerate}
		\item Semantic: does the argument preserve truth? 
		\item Syntactic: Is there a deduction for the conclusion from the premises stated in the argument? 
	\end{enumerate}

	Both of these seem like sensible perspectives to weigh an argument from. 
	
	\vspace{0.5cm} 
	
	{\bf Question:} Do we have to pick one? Which one do we use?
\end{frame}

\begin{frame}
	\frametitle{Soundness}
	
	In previous lectures we proved that the deduction system developed is \emph{sound}. 
	
	\vspace{0.5cm}
	
	\quad {\bf Theorem:} If $\Sigma \vdash \alpha $, then $\Sigma \vDash \alpha $
	
	\vspace{0.5cm}
	
	That is to say, any deduction we give must be truth preserving. Put another way, we can only provide proofs for truth preserving arguments. 
	
\end{frame}

\begin{frame}
	\frametitle{Completeness}
	
	We also proved the converse of this statement. 
	
	\vspace{0.5cm}
	
	\quad {\bf Theorem:} If $\Sigma \vDash \alpha $, then $\Sigma \vdash \alpha $
	
	\vspace{0.5cm}
	
	If an argument is truth preserving, then there is, necessarily, a deduction for it in our classical deduction system.
\end{frame}

\begin{frame}
	\frametitle{Semantic = Syntactic}
	
	Thus we do not need to choose one method or the other to decide whether an argument is good. The syntactic and semantic conclusions, as develpoed above, are equivalent notions. 
	
	\vspace{0.5cm}
	
	They judge the same arguments to be correct.  
\end{frame}

\section{Examples}

\begin{frame}
	\frametitle{Example}
	
	Consider the argument consisting of premises $\Sigma = \{\lnot (A \rightarrow B)\}$ and conclusion $\gamma = A \land B$. 
	
	\vspace{0.5cm} 
	
	\indent {\bf Question:} Is this argument valid? 
	
	\vspace{7cm}
	
\end{frame}

\begin{frame}
	\frametitle{Example continued}
	
	{\bf Question:} Given what we deduced on the previous slide, what can we say about any deductions of $\gamma$ from $\Sigma$?
	
	\vspace{7cm}
	
\end{frame}

\begin{frame}
	\frametitle{Example}
	
	Show that the following semantic consequence holds: $\alpha \lor \beta \vDash \beta \lor \alpha$
	
	\vspace{7cm}
	
\end{frame}

\begin{frame}
	\frametitle{Example continued}
	
	{\bf Question:} Given the conclusion of the previous slide, what can you deduce about any possible deductions of $\beta \lor \alpha$ from the hypothesis $\alpha \lor \beta$?
	
	\vspace{7cm}
	
\end{frame}

\begin{frame}
	\frametitle{Example}
	
	Is there a deduction of $(\alpha \lor \beta) \land \gamma$ from $\alpha \lor (\beta \land \gamma)$?
	
	\vspace{7cm}
	
	% Not you only need the second row of the truth table to show there is counterexample. 
	
\end{frame}

\begin{frame}
	\frametitle{Know the Key Definitions}
	
	Give examples of the following: 
	
	\begin{enumerate}
		\item Tautology, 
		\item contradiction, 
		\item a WFF that is neither a tautology nor a contradiction,
		\item an example of a pair of logically equivalent wff.
	\end{enumerate}

	\vspace{1cm}

	Explain why your example satisfies the stated definition. 
	
\end{frame}

\begin{frame}
	\frametitle{Know the Key Definitions}
	
	Prove that, if $\Sigma \vDash \alpha$ and $\Sigma \vDash \beta$, then $\Sigma \vDash \alpha \land \beta$
	
	\vspace{7cm}
	
\end{frame}

\begin{frame}
	\frametitle{Example Derivation}
	
	Provide a derivation for the following theorem: 
	
	$$ \ \vdash \ (\lnot \phi \rightarrow \lnot \psi) \rightarrow (\psi \rightarrow \phi)$$
	
	\vspace{7cm}
	
\end{frame}

\begin{frame}
	\frametitle{Further Reading}
	
	Here are some recommended reading to follow up on the lecture content. They are all freely available online. 
	
	\vspace{0.5cm}
	
	\begin{itemize}
		\item L$\exists \forall$N, \emph{Logic and Proof}. Sections: PL sections.
		\item Logic Matters, \emph{An Introduction to Formal Logic}: PL sections.
	\end{itemize}
	
\end{frame}





\end{document}
