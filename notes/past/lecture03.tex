% Load required themes and packages.
\documentclass{beamer}
\usetheme{Pittsburgh}
\useinnertheme[shadow]{rounded}
\setbeamercolor{block title}{bg=blue!30, fg=black}
\setbeamercolor{block body}{bg=blue!20}
\usecolortheme{default}
\useinnertheme{default}
\useoutertheme{default}
\usefonttheme{structurebold}
% Math related packges.
\usepackage{amsmath}
\usepackage{cancel}

% Natural Deduction package
\usepackage{proof}

% Remove navigation bar
\beamertemplatenavigationsymbolsempty

% Bibliography
% \usepackage[style=verbose]{biblatex}
% \bibliography{deltaBib.bib}
% Command line sequence:
% pdflatex *filename*
% biber *filename*
% pdflatex *filename*

\newtheoremstyle{indentDefn}
{\topsep} % Space above
{\topsep} % Space below
{\it} % Body font
{2cm} % Indent amount
{\bf} % Theorem head font
{:} % Punctuation after theorem head
{0.5em} % Space after theorem head
{} % Theorem head spec

\theoremstyle{indentDefn} \newtheorem{defn}[]{Definition}

\title{Lecture 3: Truth and Validity}
\author{MATH230}
\institute{Te Kura P\=angarau $\vert$ School of Mathematics and Statistics \\ Te Whare W\=ananga o Waitaha $\vert$ University of Canterbury}
\date{}

% Document body starts here.
\begin{document}


% Title frame
\begin{frame}

  \titlepage

\end{frame}

% Table of contents page
\begin{frame}
  \frametitle{Outline}

  \tableofcontents

\end{frame}

\section{Example}
	
\begin{frame}
  \frametitle{Example}
	
	``Thin is guilty," observed Watson, ``because either Holmes is right and the vile Moriarty is guilty, or he (Holmes) is wrong and Thin did the job; but those scoundrels are either both guilty or both innocent; and, as usual, Holmes is correct".
	
	\vspace{5.5cm}


	\footnotesize{Example from Richard Jeffrey, \emph{Formal Logic: Its Scope and Limits}}

\end{frame}

\section{Argument}

\begin{frame}
	\frametitle{Argument Structure}
	\begin{center}
		$\begin{array}{ l c r }			
			& \textnormal{Proposition 1} & \\
			& \textnormal{Proposition 2} & \\
			& \vdots & \\
			& \textnormal{Proposition n} & \\
			\cline{1 - 3} 
			& \textnormal{Conclusion} & 	
		\end{array}$
	\end{center}

	\vspace{1cm}

{\bf Question:} What makes for a ``good argument"? What might we mean by a ``good argument"?
\end{frame}

\section{Truth Values}

\begin{frame}
	\frametitle{Truth Values}
	
	It may either be the case that an atomic formula is true, or false.
	\begin{center}
		$\begin{array}{ l c r }			
			A & & A\\
			\cline{1 - 1} \cline{3 - 3}
			T & & 1\\ 
			F & & 0
		\end{array}$
	\end{center}	
	
	There are precisely two truth values in \emph{classical} propositional logic and each proposition must take \emph{exactly one} of them: there are no contradictions and there are no other logical values. Note that there are other logics which relax these conditions.

	
\end{frame}

\begin{frame}
	\frametitle{Truth table: ($\lnot$) Negation}
	
		\begin{center}
			$\begin{array}{ l | c r }			
				A & & \lnot A\\
				\cline{1 - 3}
				T & & \\ 
				F & & 
			\end{array}$
	\end{center}

The truth of compound variables (wff with syntactic structure) will depend, ultimately, on the truth values of the atomic formulae from which it is built.
	
\end{frame}

\begin{frame}
	\frametitle{Truth table: ($\land$) Conjunction}
	
	\begin{center}
		$\begin{array}{ l | c | r }			
		A & B & A \land B\\
		\cline{1 - 3}
		T & T & \\ 
		T & F & \\
		F & T & \\
		F & F & 	
		\end{array}$
	\end{center}

\end{frame}

\begin{frame}
	\frametitle{Truth table: ($\lor$) Disjunction}
	
	\begin{center}
		$\begin{array}{ l | c | r }			
		A & B & A \lor B\\
		\cline{1 - 3}
		1 & 1 & \\ 
		1 & 0 & \\
		0 & 1 & \\
		0 & 0 & 	
		\end{array}$
	\end{center}

	\pause
	\vspace{2cm}
	Your intuition may not agree with this. Perhaps $\lor$ should not be true when both $A$ and $B$ are true. We have another binary connective to associate to this meaning: \emph{exclusive disjunction} and we denote it by $\oplus$ or $\veebar$.

\end{frame}

\begin{frame}
	\frametitle{Truth table: ($\rightarrow$) Implication}
	
	\begin{center}
		$\begin{array}{ l | c | r }			
		A & B & A \rightarrow B\\
		\cline{1 - 3}
		1 & 1 & \\ 
		1 & 0 & \\
		0 & 1 & \\
		0 & 0 & 	
		\end{array}$
	\end{center}

	\pause
	\vspace{2cm}	
	Again, your intuition may resist here. Implication is used in this sense in mathematics, so we stick to it. See the reading at the end for a discussion on the different meanings of implication. 

\end{frame}

\begin{frame}
	\frametitle{Example}
	
	Consider the wff $(A\rightarrow(B \land A))$ that involves more than one binary operation. 
	
	\begin{center}
		$\begin{array}{ c | c | c | c }			
			A & B & B \land A & (A\rightarrow(B \land A))\\
			\cline{1 - 4}
			1 & 1 & & \\ 
			1 & 0 & &\\
			0 & 1 & &\\
			0 & 0 & &	
		\end{array}$
	\end{center}	
\end{frame}

\section{Valid Arguments}

\begin{frame}
	\frametitle{Valid Argument}
	
	Suppose $\Sigma$ is the set of premises of an argument with $\gamma$ as the conclusion.
	
	\begin{defn} We say an argument is valid if its conclusion is true in every case in which each of its premises are true.
	\end{defn}
 
	\vspace{0.5cm}
	
	If the argument for $\gamma$ from $\Sigma$ is valid, then we say $\gamma$ is a \emph{semantic consequence of } $\Sigma$ and denote it $\Sigma \vDash \gamma$ 
	
	\vspace{1cm}
	

\end{frame}

\begin{frame}
	\frametitle{Example: Modus Ponens}
	
	Show $\{A, (A \rightarrow B)\} \vDash B$
	
	
	\pause
	\vspace{1cm}
	\begin{center}
		$\begin{array}{ c | c | c }			
			A & B & (A\rightarrow B)\\
			\cline{1 - 3}
			1 & 1 & 1 \\ 
			1 & 0 & 0\\
			0 & 1 & 1\\
			0 & 0 & 1	
		\end{array}$
	\end{center}	

	\pause
	\vspace{1cm}
	There is only one case in which both of the premises are true. In that case, the conclusion is also true.
		\begin{center}
			$\begin{array}{ c | c | c }			
				A & B & (A\rightarrow B)\\
				\cline{1 - 3}
				1 & 1 & 1 \\ 	
			\end{array}$
		\end{center}	
	
\end{frame}

\begin{frame}
	\frametitle{Counterexamples}
	
	\begin{defn}[Counterexample] A counterexample to an argument is a case in which the premises are all true, but the conclusion is false. \end{defn}
	
	%\vspace{0.5cm}	
	
	{\bf Example: Affirming the Consequent} 
	
	Show $\{B, (A \rightarrow B)\} \nvDash A$
	
	\begin{center}
		$\begin{array}{ c | c | c }			
			A & B & (A\rightarrow B)\\
			\cline{1 - 3}
			1 & 1 & 1 \\ 
			1 & 0 & 0\\
			0 & 1 & 1\\
			0 & 0 & 1	
		\end{array}$
	\end{center}

This time the one line of interest is the counterexample: $A$ (the conclusion) is false, but the premises are true. 
	
	\begin{center}
		$\begin{array}{ c | c | c }			
			A & B & (A\rightarrow B)\\
			\cline{1 - 3}
			0 & 1 & 1 \\ 	
		\end{array}$
	\end{center}	
	
\end{frame}

\begin{frame}
  \frametitle{Further Reading}

  Here are some recommended reading to follow up on the lecture content. They are all freely available online. 

	\vspace{0.5cm}

	\begin{itemize}
		\item L$\exists \forall$N, \emph{Logic and Proof}. Sections: 3.1, 6.1, and 6.2 
		\item Logic Matters, \emph{An Introduction to Formal Logic}: p172 - 216
	\end{itemize}

\end{frame}





\end{document}
