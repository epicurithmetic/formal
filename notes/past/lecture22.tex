% Load required themes and packages.
\documentclass{beamer}
\usetheme{Pittsburgh}
\usecolortheme{default}
\useinnertheme{default}
\useoutertheme{default}
\usefonttheme{structurebold}
% Math related packges.
\usepackage{amsmath}
\usepackage{cancel}

% Natural Deduction package
\usepackage{proof}

% Package for hyperlinks
\usepackage{hyperref}

% Package for centering quotes
\usepackage{csquotes}

% Remove navigation bar
\beamertemplatenavigationsymbolsempty

% Bibliography
% \usepackage[style=verbose]{biblatex}
% \bibliography{deltaBib.bib}
% Command line sequence:
% pdflatex *filename*
% biber *filename*
% pdflatex *filename*

% Turing Machine macros
\usepackage{tikz}
\def \bcell {\hspace{0.5em}}
\def \tmtape#1#2#3#4#5#6#7#8#9 {
	$\begin{array}{|||c|c|c|c|c|c|c|c|c|c|c|c|c|c|c|c|c|c}
		\cline{1-17}
		@ & 
		#1 &
		#2 & 
		#3 & 
		#4 & 
		#5 & 
		#6 & 
		#7 & 
		#8 &
		#9 & 
		\bcell & 
		\bcell &
		\bcell & 
		\bcell & 
		\bcell &
		\bcell & 
		\bcell & 
		\cdots  \\
		\cline{1-17}
	\end{array}$
	}

\def \headposition#1 {
	% Gross hack to get the head position indicator lined up with the cell set by the input of this function.
	\ifnum#1=1 \hspace{0.1cm} $\bigtriangleup$ \else 
		\ifnum#1=2 \hspace{0.7cm} $\bigtriangleup$ \else 
			\ifnum#1=3 \hspace{1.25cm} $\bigtriangleup$ \else  
				\ifnum#1=4 \hspace{1.8cm} $\bigtriangleup$ \else 
					\ifnum#1=5 \hspace{2.35cm} $\bigtriangleup$ \else 
						\ifnum#1=6 \hspace{2.9cm} $\bigtriangleup$ \else 
							\ifnum#1=7 \hspace{3.45cm} $\bigtriangleup$ \else 
								\ifnum#1=8 \hspace{4cm} $\bigtriangleup$ \else 
									\ifnum#1=9 \hspace{4.55cm} $\bigtriangleup$ \else
										\ifnum#1=10 \hspace{5.1cm} $\bigtriangleup$ \else  \fi
									
									\fi								
								\fi
							\fi
						\fi
					\fi				
				\fi			
			\fi		
		\fi
	 \fi}

\title{Lecture 22: Turing Machines}
\author{MATH230}
\institute{Te Kura P\=angarau $\vert$ School of Mathematics and Statistics \\ Te Whare W\=ananga o Waitaha $\vert$ University of Canterbury}
\date{}

% Document body starts here.
\begin{document}


% Title frame
\begin{frame}

  \titlepage

\end{frame}

% Table of contents page
\begin{frame}
  \frametitle{Outline}

  \tableofcontents

\end{frame}

\section{On Computable Numbers}

\begin{frame}
	\frametitle{Summary}

	In 1931 G\"{o}del proved his incompleteness theorems. These gave \emph{negative} answers to some of the metaquestions that Hilbert had asked. The question of decidability remained open. Negative answers would have been easier to entertain in the light of G\"{o}del.

	\vspace{0.5cm}

	Although important for the development of the theory of computation, primitive recursion and $\mu$-recursion were not used to resolve the effective calculability metaquestions. It was papers on other definitions of computation that settled the entscheidungsproblem and the undecidability of arithmetic.

	\vspace{0.5cm}

	Both papers appeared in 1936 at about the same time: 

	\begin{enumerate}
		\item[] Turing Machines (Alan Turing)
		\item[] $\lambda$-calculus (Alonso Church)
	\end{enumerate}

	Turing went to do his PhD with Church \emph{after} these were published. 
	
\end{frame}

\begin{frame}
  \frametitle{Alan Turing, 1936}

	In 1936 Alan Turing published the paper \emph{On Computable numbers, with an application to the entscheidungsproblem.} In this paper he presented his reply to the hope for a definition of an effective procedure.

	\vspace{0.5cm}

	He developed the theory of $\mathfrak{a}$-machines and used them to answer the entscheidungsproblem in the negative. 

	\vspace{0.5cm}

	Turing called his machines automatic ($\mathfrak{a}$) machines. It was his thesis advisor, Church, that coined the term Turing Machine. 



\end{frame}

\begin{frame}
	\frametitle{Finite Means}

	Turing opened his paper as follows: 

	\begin{displayquote}
		Computable numbers may be described briefly as the real numbers whose expressions as a decimal are calculable by finite means. 
	\end{displayquote}
	
	Turing did not mean to say the number itself had to be expressed as a finite decimal. But that the means for calculating it were finitely expressed. 

	\vspace{0.5cm}

	This sets his perspective apart from that of primitive recursion (and $\lambda$-calculus) by focussing on the number, rather than functions on numbers. He quickly points out that it is just a change in perspective and that his approach can equally be considered from the point-of-view of functions. 

\end{frame}

\begin{frame}
	\frametitle{Step-by-Step Calculation}

	\begin{displayquote}
		We may compare a man in the process of computing a real number to a machine which is only capable of a finite number of conditions $q_{1}, q_{2}, \dots , q_{R}$ which will be called $m$-configurations.
	\end{displayquote}

	\vspace{0.5cm}

	\begin{tabular}{c c}

		\hspace{1cm}

		\begin{tabular}{c c c c c c}

			1 & 0 & 1 & 1 & 0 & 1 \\
			  &   &   & + & 1 & 0 \\
			\hline
			
		\end{tabular}

		\hspace{1cm}

		& 

		%\hspace{1cm}

		\begin{tabular}{ c | c c }

			$+$ & 0 & 1 \\
			\hline
			0 & 0 & 1\\
			1 & 1 & 0
			
		\end{tabular}
		
	\end{tabular}

	\vspace{5cm}
	
\end{frame}

\begin{frame}
	\frametitle{$a$-Machines}

	\begin{displayquote}

		The machine is supplied with a ``tape" (the analogue of paper) running through it, and divided into sections (called ``squares") each capable of bearing a ``symbol''
		
	\end{displayquote}

	\vspace{0.3cm}

 	\tmtape{1}{0}{1}{1}{0}{1}{0}{0}{1} \\
 	\headposition{1}	

	\vspace{5cm}
	
\end{frame}

\section{Turing Machines}

\begin{frame}
	\frametitle{Turing Machine}

		{\bf Tape}

		\begin{itemize}
			\item[] Tape has a start (@) cell. 
			\item[] Tape is infinite to the right. 
			\item[] Tape squares hold all input, output, and working.
		\end{itemize}

		{\bf Read Head}

		\begin{itemize}
			\item[] Scan one square at a time; scanned square.
			\item[] Can move left, nowhere, right (L/N/R).
			\item[] Prints symbols to a square.
		\end{itemize}

		{\bf $\mathfrak{m}$-configurations} ($q_{1}, q_{12}, q_{13}$ etc.)

		\begin{itemize}
			\item[] Finitely many configurations.
			\item[] Print/Move/Update all defined by configurations.
			\item[] How to react to the scanned symbol.
			\item[] Start in $q_{0}$. Special \emph{HALT} state.
		\end{itemize}
\end{frame}

\begin{frame}
	\frametitle{Example: Boring}
	
	\tmtape{\bcell}{\bcell}{\bcell}{\bcell}{\bcell}{\bcell}{\bcell}{\bcell}{\bcell} \\
	\headposition{1}
	
	\begin{tabular}{lr}
		
		$\begin{array}{ c | c | c | c | c }
		C & R & P & M & U \\
		\cline{1-5}
		q_{0} & @ & @ & R & H
		\end{array}$
		
		\hspace{1.5cm}
		
		&
				
		\begin{tabular}{rl}
			\bcell \\ % blank line for formatting
			\bcell \\ % blank line for formatting			
			{\bf Current State:} & $q_{0}$\\ % Perhaps leave this out to fill in during the lecture. 
			{\bf Scanned Symbol:} & $@$\\
			{\bf Printed Symbol:} & $@$\\
			{\bf Move Direction:} & $R$\\
			{\bf Update State:} & $H$								
			
		\end{tabular}
		
	\end{tabular}
	\vspace{3cm}
	
	
\end{frame}

%%%
%%%	Example with printing 1,0,1,0,1,0,1,0,...
%%%

% Step 0
\begin{frame}
	\frametitle{Example: All Ones n Zeroes}

	Write instructions for a Turing machine that will not HALT and prints an alternating pattern of 1s and 0s.

	\vspace{0.3cm}
	
	\tmtape{\bcell}{\bcell}{\bcell}{\bcell}{\bcell}{\bcell}{\bcell}{\bcell}{\bcell} \\
	\headposition{1}
	
	\begin{tabular}{lr}
		
		$\begin{array}{ c | c | c | c | c }
		C & R & P & M & U \\
		\cline{1-5}
		q_{0} & @ & @ & R & \ \\ 
		 \ & \ & \ & \ & \ \\
		 \ & \ & \ & \ & \\
		 \ & \ & \ & \ & 
		\end{array}$
		
		\hspace{1.5cm}
		
		&
				
		\begin{tabular}{rl}
			\bcell \\ % blank line for formatting
			\bcell \\ % blank line for formatting			
			{\bf Current State:} & $q_{0}$\\ % Perhaps leave this out to fill in during the lecture. 
			{\bf Scanned Symbol:} & $@$\\
			{\bf Printed Symbol:} & $@$\\
			{\bf Move Direction:} & $R$\\
			{\bf Update State:} & \ 								
			
		\end{tabular}
		
	\end{tabular}
	\vspace{3cm}
	
	
\end{frame}

% Step 1
\begin{frame}
	\frametitle{Example: All Ones n Zeroes}

	\tmtape{\bcell}{\bcell}{\bcell}{\bcell}{\bcell}{\bcell}{\bcell}{\bcell}{\bcell} \\
	\headposition{2}
	
	\begin{tabular}{lr}
		
		$\begin{array}{ c | c | c | c | c }
		C & R & P & M & U \\
		\cline{1-5}
		q_{0} & @ & @ & R & q_{0} \\ 
		q_{0} & b & 1 & R & q_{1} \\
		q_{1} & b & 0 & R & q_{0}
		\end{array}$
		
		\hspace{1.5cm}
		
		&
				
		\begin{tabular}{rl}
			\bcell \\ % blank line for formatting
			\bcell \\ % blank line for formatting			
			{\bf Current State:} & $q_{0}$\\ % Perhaps leave this out to fill in during the lecture. 
			{\bf Scanned Symbol:} & $b$\\
			{\bf Printed Symbol:} & $1$\\
			{\bf Move Direction:} & $R$\\
			{\bf Update State:} & $q_{1}$								
			
		\end{tabular}
		
	\end{tabular}
	\vspace{3cm}
	
	
\end{frame}

% Step 2
\begin{frame}
	\frametitle{Example: All Ones n Zeroes}

	\tmtape{1}{\bcell}{\bcell}{\bcell}{\bcell}{\bcell}{\bcell}{\bcell}{\bcell} \\
	\headposition{3}
	
	\begin{tabular}{lr}
		
		$\begin{array}{ c | c | c | c | c }
		C & R & P & M & U \\
		\cline{1-5}
		q_{0} & @ & @ & R & q_{0} \\ 
		q_{0} & b & 1 & R & q_{1} \\
		q_{1} & b & 0 & R & q_{0}
		\end{array}$
		
		\hspace{1.5cm}
		
		&
				
		\begin{tabular}{rl}
			\bcell \\ % blank line for formatting
			\bcell \\ % blank line for formatting			
			{\bf Current State:} & $q_{1}$\\ % Perhaps leave this out to fill in during the lecture. 
			{\bf Scanned Symbol:} & $b$\\
			{\bf Printed Symbol:} & $0$\\
			{\bf Move Direction:} & $R$\\
			{\bf Update State:} & $q_{0}$								
			
		\end{tabular}
		
	\end{tabular}
	\vspace{3cm}
	
	
\end{frame}

% Step 3
\begin{frame}
	\frametitle{Example: All Ones n Zeroes}

	\tmtape{1}{0}{\bcell}{\bcell}{\bcell}{\bcell}{\bcell}{\bcell}{\bcell} \\
	\headposition{4}
	
	\begin{tabular}{lr}
		
		$\begin{array}{ c | c | c | c | c }
		C & R & P & M & U \\
		\cline{1-5}
		q_{0} & @ & @ & R & q_{0} \\ 
		q_{0} & b & 1 & R & q_{1} \\
		q_{1} & b & 0 & R & q_{0}
		\end{array}$
		
		\hspace{1.5cm}
		
		&
				
		\begin{tabular}{rl}
			\bcell \\ % blank line for formatting
			\bcell \\ % blank line for formatting			
			{\bf Current State:} & $q_{0}$\\ % Perhaps leave this out to fill in during the lecture. 
			{\bf Scanned Symbol:} & $b$\\
			{\bf Printed Symbol:} & $1$\\
			{\bf Move Direction:} & $R$\\
			{\bf Update State:} & $q_{1}$								
			
		\end{tabular}
		
	\end{tabular}
	\vspace{3cm}
	
	
\end{frame}

\begin{frame}
	\frametitle{Specification and Execution}

	To specify a Turing machine is to write down the table of states. Turing eventually converges on the following standard which we will follow in this course: 
	
	\vspace{0.3cm}

	\begin{center}
		\begin{tabular}{c c}

		
			$\begin{array}{ c | c | c | c | c }
				C & R & P & M & U \\
				\cline{1-5}
				q_{0} & @ & @ & R & q_{0} \\ 
				q_{0} & b & 1 & R & q_{1} \\
				q_{1} & b & 0 & R & q_{0}
			\end{array}$


			\hspace{1cm} 

			&

			$q_{5},1,0,L,q_{10}$

		\end{tabular}
	\end{center}

	The execution of the Turing machine consists of starting at the $q_{0}$ state, carrying out the instructions of the current state (in the order presented), followed by the instructions of the following states. 

	\vspace{0.5cm}

	The execution loop is only stopped if the special HALT state is reached at some point in the execution of the machine instructions. 
	
\end{frame}

\begin{frame}
	\frametitle{Conventions}
	The following is a list of things to remember 

	\vspace{0.3cm}

	\begin{itemize}
		\item[] Each instruction has a print command.
		\item[] Each instruction has a move command. 
		\item[] Special character $b$ means \emph{blank square}.
		\item[] Always HALT on the home (@) symbol.
		\item[] Function input next to @ follow by blanks
		\item[] Function output next to @ followed by blanks
		\item[] Use non-numerical symbols for working
	\end{itemize}

	\vspace{0.5cm}

	Give a high-level description of every machine you write. 

	\vspace{0.2cm}

	Comment your machine instructions!

\end{frame}

\begin{frame}
	\frametitle{Example: Flip-bits}

	Write a Turing machine to flip the bits on a binary input. 

	\vspace{0.3cm}

	\tmtape{1}{0}{1}{1}{0}{1}{1}{\bcell}{\bcell} \\
	\headposition{1}

	\vspace{0.3cm}

	$\begin{array}{ c | c | c | c | c }
		C & R & P & M & U \\
		\cline{1-5}
		q_{0} & @ & @ & R & \ \\ 
		 \ & \ & \ & \ & \ \\
		 \ & \ & \ & \ & \\
		 \ & \ & \ & \ & \\ 
		 \ & \ & \ & \ & \\ 
		 \ & \ & \ & \ &   
	\end{array}$

	\vspace{3cm}
	
\end{frame}


\begin{frame}
	\frametitle{Example: Reverse}

	Write a Turing machine to reverse a binary input. 

	\vspace{0.3cm}

	\tmtape{1}{0}{1}{1}{0}{1}{1}{\bcell}{\bcell} \\
	\headposition{1}

	\vspace{0.3cm}

	$\begin{array}{ c | c | c | c | c }
		C & R & P & M & U \\
		\cline{1-5}
		q_{0} & @ & @ & R & \ \\ 
		 \ & \ & \ & \ & \ \\
		 \ & \ & \ & \ & \\
		 \ & \ & \ & \ & \\ 
		 \ & \ & \ & \ & \\ 
		 \ & \ & \ & \ &   
	\end{array}$

	\vspace{3cm}
	
\end{frame}


\begin{frame}
	\frametitle{Further Reading}
	
	Here are some recommended reading to follow up on the lecture content.
	
	\vspace{0.5cm}
	
	\begin{itemize}
		\item The Annotated Turing \emph{Charles Petzold}
		\item SEP: Turing Machines
	\end{itemize}
	
\end{frame}





\end{document}
