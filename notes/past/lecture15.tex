% Load required themes and packages.
\documentclass{beamer}
\usetheme{Pittsburgh}
\usecolortheme{default}
\useinnertheme{default}
\useoutertheme{default}
\usefonttheme{structurebold}
% Math related packges.
\usepackage{amsmath}
\usepackage{cancel}

% Natural Deduction package
\usepackage{proof}

% Package for hyperlinks
\usepackage{hyperref}

% Remove navigation bar
\beamertemplatenavigationsymbolsempty

% Bibliography
% \usepackage[style=verbose]{biblatex}
% \bibliography{deltaBib.bib}
% Command line sequence:
% pdflatex *filename*
% biber *filename*
% pdflatex *filename*

\title{Lecture 15: First-Order Theories of Arithmetic II}
\author{MATH230}
\institute{Te Kura P\=angarau $\vert$ School of Mathematics and Statistics \\ Te Whare W\=ananga o Waitaha $\vert$ University of Canterbury}
\date{}

% Document body starts here.
\begin{document}


% Title frame
\begin{frame}

  \titlepage

\end{frame}

% Table of contents page
\begin{frame}
\frametitle{Outline}

	\tableofcontents

\end{frame}


\section{Peano Arithmetic}
\begin{frame}
	\frametitle{Axioms of PA}
	
	Peano Arithmetic has signature PA$: \{0,s,+,\times\}$ and axioms
	
	\begin{enumerate}
		\item $\forall x \lnot(s(x) = 0)$
		\item $\forall x \ \forall y ((s(x) = s(y)) \to (x = y))$
		%\item $\forall x \ ((x = 0) \lor \exists y (x = s(y)))$
		\item $\forall x \ (x + 0 = x)$
		\item $\forall x \ \forall y \ (x + s(y) = s(x + y))$
		\item $\forall x \ (x \times 0 = 0)$
		\item $\forall x \ \forall y \ (x \times s(y) = (x \times y) + x)$
		\item $(P(0) \land \forall x \ (P(x) \to P(x+1))) \rightarrow \forall y (P(y))$
	\end{enumerate}

	\vspace{0.2cm} 
	
	This is Presburger arithmetic with multiplication. It is accepted as \emph{the} first-order theory of arithmetic.
	
\end{frame}

\begin{frame}
	\frametitle{Abbreviations}
	
	We make the following abuse of notation 
	\begin{center}
	\begin{itemize}
		\item[] $1 = s(0)$
		\item[] $2 = s(1) = s(s(0))$
		\item[] $3 = s(2) = s(s(1)) = s(s(s(0)))$
		\item[] $\vdots$
	\end{itemize}
	\end{center}
	
	It is important to distinguish well-formed formulae in our formal language from statements in the intended model. Remember the well-formed formulae in our language are just strings of symbols. Some authors force this with their notation; we will just be mindful of the difference. 
	
\end{frame}

\begin{frame}
	\frametitle{Example}
	
	$$ PA \vdash 2 + 1 = 3$$
	
	\vspace{7cm}
	
	
\end{frame}

\begin{frame}
	\frametitle{Example}
	
	$$ PA \vdash 2 \neq 1$$

	\vspace{7cm}
	
	
\end{frame}

\begin{frame}
	\frametitle{Example}
	
	$$ PA \vdash 0 + 1 = 1$$
	
	\vspace{7cm}
	
\end{frame}

\begin{frame}
	\frametitle{Induction Example}
	
	$$ PA \vdash \forall x \ (0 + x = x)$$
	
	\vspace{7cm}
	
\end{frame}

\begin{frame}
	\frametitle{Example}
	
	$$ PA \vdash 7 \times 1 = 7$$
	
	\vspace{7cm}
	
\end{frame}


\section{Theorem Provers and Proof Checkers}

\begin{frame}
	\frametitle{Is This Feasible?}
	
	% Make some comments about computers, and proof generators and checkers.
	
	We appear to have a precise formal language in which we can prove real number theoretic statements - great! 
	
	\vspace{0.2cm}
	
	Unfortunately, it takes slides and slides to prove basic statements about arithmetic - let alone something non-trivial like the fundamental theorem of arithmetic. 
	
	\vspace{0.7cm}
	
	Happily we now have computers! These computers can store sub-proofs to make writing new proofs much easier. The \textbf{Xena Project} is building a database of undergraduate math with proofs verified by the proof checker L$\exists \forall$N. 
	
	\vspace{0.2cm}
	
	Tutorial 6 will have you work in L$\exists\forall$N. This way you get an idea for how these ideas have had an effect today, on modern mathematics. Plus, proving basic properties like commutativity of addition is much nicer in L$\exists \forall$N, than writing it on paper. 
\end{frame}

\section{Answering Hilbert}

\begin{frame}
	\frametitle{Non-Standard Models}
	
	There are non-standard models of Peano Arithmetic! Thoralf Skolem and Abraham Robinson are the pioneers of non-standard models. 
	
	\vspace{0.5cm}
	
	{\bf Example:} One can make a model with an element, $x$, larger than every natural number. (Consequence of the compactness theorem).
	
	\vspace{0.5cm}
	
	{\bf Example:} Hyperreal numbers (the hypernaturals) were constructed by Abraham Robinson in the 1960s. These provide another non-standard model of Peano Arithmetic. 
	
	\vspace{0.5cm}
	
	This already suggests that Peano Arithmetic is \emph{incomplete}.
	
\end{frame}

\begin{frame}
	\frametitle{G\"{o}del's First Incompleteness Theorem}
	
	In 1931 G\"{o}del published the following theorem:
	
	\vspace{0.5cm}
	
	Any consistent formal theory $F$ within which Peano arithmetic can be carried out, is necessarily incomplete. That is, there are statements, $\phi$, in $F$ such that neither $F \vdash \phi$ nor $F \vdash \lnot \phi$. 
	
	\vspace{0.5cm} 
	
	Thus, Peano arithmetic is incomplete. 
	
	\vspace{0.5cm}
	
	Perhaps we can add axioms to PA to make it complete? No!  
	
	\vspace{0.5cm} 
	
	{\bf Peano arithmetic is incompletable!}
	
\end{frame}

\begin{frame}
	\frametitle{G\"{o}del's Second Incompleteness Theorem}
	
	In 1931 G\"{o}del published the following theorem:
	
	\vspace{0.5cm}
	
	For any consistent formal theory $F$ within which Peano arithmetic can be carried out, the consistency of $F$ cannot be proved in $F$ itself. 
	
	\vspace{0.5cm} 
	
	Thus, Peano arithmetic can't prove it's own consistency. Nor can any extension of it. 
	
	\vspace{0.5cm} 
	
	G\"{o}del showed the consistency of PA can be written in PA and is one of those G\"{o}delian sentences!
	
\end{frame}

\section{Robinson Arithmetic}

\begin{frame}
	\frametitle{Axioms of Q}
	
	Robinson Arithmetic has signature $\mathcal{Q}: \{0,s,+,\times\}$ and axioms
	
	\begin{enumerate}
		\item $\forall x \lnot(s(x) = 0)$
		\item $\forall x \ \forall y ((s(x) = s(y)) \to (x = y))$
		\item $\forall x \ ((x = 0) \lor \exists y (x = s(y)))$
		\item $\forall x \ (x + 0 = x)$
		\item $\forall x \ \forall y \ (x + s(y) = s(x + y))$
		\item $\forall x \ (x \times 0 = 0)$
		\item $\forall x \ \forall y \ (x \times s(y) = (x \times y) + x)$
	\end{enumerate}
	
	\vspace{0.2cm}
	
	Without the induction schema $\mathcal{Q}$ is \emph{fintely} axiomatized, and much weaker than Peano arithmetic. However, Raphael Robinson introduced $\mathcal{Q}$ in 1950 to show that although $\mathcal{Q}$ is much weaker than PA, it (and any extension of $\mathcal{Q}$) is still incomplete and undecidable. There is enough number theory to carry out G\"{o}del's proof. 
	
	% It's not the infinite number of axioms that cause the incompleteness or undecidability.
	
\end{frame}

\begin{frame}
	\frametitle{Further Reading}
	
	Here are some recommended reading to follow up on the lecture content. They are all freely available online. 
	
	\vspace{0.5cm}
	
	\begin{itemize}
		\item L$\exists \forall$N, \emph{Logic and Proof}. Sections: 17, 18 (+ 23 optional).
		\item Logic Matters, \emph{G\"{o}del Without Too Many Tears}.
		\item SEP, \emph{G\"{o}del's Incompleteness Theorem} (Optional!).
	\end{itemize}
	
\end{frame}

\end{document}
