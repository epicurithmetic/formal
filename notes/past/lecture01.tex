% Load required themes and packages.
\documentclass{beamer}
\usetheme{Pittsburgh}
\usecolortheme{default}
\useinnertheme{default}
\useoutertheme{default}
\usefonttheme{structurebold}
% Math related packges.
\usepackage{amsmath}

% Bibliography
% \usepackage[style=verbose]{biblatex}
% \bibliography{deltaBib.bib}
% Command line sequence:
% pdflatex *filename*
% biber *filename*
% pdflatex *filename*

% Hyperlinks
\usepackage{hyperref}
\hypersetup{linkcolor=blue}

% Remove navigation bar
\beamertemplatenavigationsymbolsempty

\newtheoremstyle{indentDefn}
{\topsep} % Space above
{\topsep} % Space below
{\it} % Body font
{2cm} % Indent amount
{\bf} % Theorem head font
{:} % Punctuation after theorem head
{0.5em} % Space after theorem head
{} % Theorem head spec

\theoremstyle{indentDefn} \newtheorem{defn}[]{Definition}

\title{Lecture 1: Welcome}
\subtitle{Logic, Automata, and Computability}
\author{MATH230}
\institute{Te Kura P\=angarau $\vert$ School of Mathematics and Statistics \\ Te Whare W\=ananga o Waitaha $\vert$ University of Canterbury}
\date{}


% Document body starts here.
\begin{document}


% Title frame
\begin{frame}

  \titlepage

\end{frame}

% Table of contents page
\begin{frame}
  \frametitle{Outline}

  \tableofcontents

\end{frame}

\section{Introduction}



\begin{frame}
  \frametitle{Welcome to MATH230}
	
	\begin{itemize}
		\item[]	Lecturer: Robert Culling (Rob)\\
			\quad Office: Jack Erskine 422\\
			\quad Email: robert.culling@canterbury.ac.nz
			
			\vspace{0.1cm}
			
			\quad Feel free to come and chat about the course anytime. 
			
			\vspace{0.2cm}
	
	
		\item[] Tutor: Kerry Manson
		
			\quad Tutorials are on Monday evenings every week. 
			
			\quad Hand-in select problems that Wednesday.
	
			\vspace{0.2cm}
	
		\item[]	Class Representative: ? Could be you ?
		
		% Be sure to get an idea of the cohort. What do they study? Major?
	
	\end{itemize}
\end{frame}

\begin{frame}
	\frametitle{Who is taking this course?}
	
	What is your major? Why have you taken this course?
	
	\vspace{0.5cm}
	
	What do you hope to get out of this course? 
	
\end{frame}

\begin{frame}
	\frametitle{What Will We Study?}
	
	Mathematics, and metamathematics, that came into focus in the 1800s and blossomed in the 1900s. 
	
	\vspace{0.2cm}
	
	Work that was motivated by the following questions about mathematics: 
	
	\begin{itemize}
		\item Is mathematics consistent?
			% There were concerns from some of the work being done at the time: multiple geometries, new types of numbers, contradictions in the theory of sets. 		
		\item Can we determine/prove all true statements?
		\item What does it mean to give a proof?
		\item Is there a fixed process/procedure/algorithm that we can give a statement to tell us whether or not the statement is true?
	\end{itemize}
	
	{\bf Hilbert's Program} called for an axiomatic formalisation of mathematics so that these questions \textit{about} mathematics could be answered precisely. 
	
\end{frame}

\section{Course Overview}

\begin{frame}
	\frametitle{What Will We Study?}
	
	In order to be precise about these issues mathematicians chose to write mathematics in a precise language: first order predicate logic. 
	
	\vspace{0.2cm}
	
	With mathematics written in a particular language it is easier to say what exactly we mean by a proof. 
	
	\vspace{0.2cm} 
	
	Furthermore the precise language is able to be encoded into numerals. This enabled mathematicians to write down specifically what one might mean by ``an effective procedure". Thus mathematicians could get into a position to give answers to the possibility of a ``universal proof procedure".
	
\end{frame}

\begin{frame}
	\frametitle{What Will We Study?}
	
	As we follow this story we will talk about the following topics: 
	
	\vspace{0.2cm}
	
	\begin{itemize}
		\item Propositional logic.
		\item First order logic.
		\item First order theories of arithmetic.
		\item Theories of computation.
		\item Limits of computation and mathematics. 
	\end{itemize}
\end{frame}

\begin{frame}
	\frametitle{Contemporary Influence}
	
	This subject, the foundations of mathematics, lives on today in many different forms. For example 
	
	\begin{itemize}
		\item Research of mathematicians from around the world.
		\item Philosophy.
		\item Interactive proof checkers.
		\item Computers and computer science. 
	\end{itemize}

	In particular researchers at Microsoft have developed the open source project L$\exists \forall$N. This is an interactive theorem prover. 
	
	\vspace{0.2cm}
	
	This project boasts an ever growing library of definitions, theorems, and proofs of mathematics in a formal language. Your first tutorial will introduce you to this programming language via the	\href{https://www.ma.imperial.ac.uk/~buzzard/xena/natural_number_game/}{Natural Numbers Game}.

\end{frame}

\begin{frame}
	\frametitle{Report}
	
	We can only cover so much in lectures. These topics have a vast history and many interesting contemporary ideas. As part of your assessment you will be writing a report on some idea, person, machine, model, logic, that we just don't have time to cover. 
	
	\vspace{0.1cm} 
	
	\begin{itemize}
		\item Logic 
				\begin{itemize}
					\item Non-standard logics 
					\item Different proof theories 
					\item Different semantics
					\item Philosophy
				\end{itemize}
		\item Models of Computation
				\begin{itemize}
					\item Elementary cellular automata
					\item Conway's game of Life 
					\item Register machines
					\item Ada Lovelace and the first computer program
				\end{itemize}
		\item People, changing of ideas, theorem checkers/provers etc.  
		
	\end{itemize}

	Come and chat with me to finalise your topic. 
	
	
\end{frame}

\begin{frame}
	\frametitle{Assessment}
	
	\begin{itemize}
		\item Report 20\%
		\item Tutorials 10\% (Hand-in 10 out of the 12).
		\item Test 15\%.
		\item Assignments (Turing machines) 20\%.
		\item Examination 35\% (With 40\% hurdle).
	\end{itemize}
	
	\vspace{0.2cm}
	
	All official course information can be found on Learn.
	
\end{frame}

\begin{frame}
	\frametitle{Expectations}
	
	University guidelines suggest spending at least 10 hours per week on a 15 point course. 
	
	\begin{itemize}
		\item 3 hours on lectures actively taking notes 
		
		(Double-speed is not good!)
		\item 1 hour in tutorial
		\item 6 hours extra to study. What can you do? 
	\end{itemize}
	% Make a note on reading mathematics - you won't get everything straight away, but that's okay. It will come together. 
	\vspace{3cm}
	
	{\bf Expectations of me:} You can expect me to be available to help in my office in appointed times or when we have planned a meeting. I want you to do well and I will make sure I am here to help that happen. 
	
\end{frame}

\section{Recommended Resources}
\begin{frame}
  \frametitle{Further Reading}

  Throughout the course I will recommend you read certain sections of two texts: both of which are freely available online at the following links. 
  
  \vspace{0.5cm}
  
  \begin{itemize}
  	\item L$\exists \forall$N: Logic and Proof. \\ 
  			@ https://leanprover.github.io/logic\_and\_proof/index.html
  	\item Logic Matters: An Introduction to Formal Logic. \\ 
  			@ https://www.logicmatters.net/ifl/  	
   	\item The Stanford Encyclopedia of Philosophy.
  \end{itemize}

	\vspace{0.5cm}

	I encourage you to find your own extra resources online and in the libraries on campus. However, beware that formal logic has a lot of symbols and authors don't always agree on which symbols to use for which concepts. So, notation may differ from ours. Always use our notation in assessments. 

\end{frame}





\end{document}
