% Load required themes and packages.
\documentclass{beamer}
\usetheme{Pittsburgh}
\usecolortheme{default}
\useinnertheme{default}
\useoutertheme{default}
\usefonttheme{structurebold}
% Math related packges.
\usepackage{amsmath}
\usepackage{cancel}
\usepackage{soul}

% Natural Deduction package
\usepackage{proof}

% Package for hyperlinks
\usepackage{hyperref}

% Remove navigation bar
\beamertemplatenavigationsymbolsempty

% Bibliography
% \usepackage[style=verbose]{biblatex}
% \bibliography{deltaBib.bib}
% Command line sequence:
% pdflatex *filename*
% biber *filename*
% pdflatex *filename*

\title{Lecture 25: $\lambda$ Calculus}
\author{MATH230}
\institute{Te Kura P\=angarau $\vert$ School of Mathematics and Statistics \\ Te Whare W\=ananga o Waitaha $\vert$ University of Canterbury}
\date{}

% Document body starts here.
\begin{document}


% Title frame
\begin{frame}

  \titlepage

\end{frame}

% Table of contents page
\begin{frame}
  \frametitle{Outline}

  \tableofcontents

\end{frame}

\section{Lambda ($\lambda$) Calculus}

\begin{frame}
	\frametitle{Lambda Calculus}

	The lambda calculus is a formal language with rules of formatation, manipulation, and simplification of strings called $\lambda$-expressions. 

	\vspace{0.5cm}

	It is the process of ``simplification" - to be known as $\beta$-reduction - that is the process of computation in this model.
	
	\vspace{0.5cm}

	Some of the ideas of the lambda calculus go back to Gottlob Frege, but Alonso Church took those ideas and developed the theory proper through a series of papers in the 1930s. 

	\vspace{0.5cm}

	This (with its type theory counterparts) has fascinating links to proofs in first-order logic and through this to program and proof verification.

\end{frame}

\begin{frame}
	\frametitle{Grammar of Lambda Calculus}

	Lambda calculus has countably many variables $x,y,z,a,b,c,\dots$

	\vspace{0.5cm}

	$\lambda$-expressions are constructed from variables using the following three mechanisms.

	\begin{itemize}
		\item[] Variables: $x,y,z, \dots$
		\item[] Application: $(t \ u)$ for $\lambda$-expressions $t,u$
		\item[] Abstraction: $(\lambda x. \ t)$ for variable $x$ and $\lambda$-expression $t$
	\end{itemize}

	\vspace{0.2cm}

	{\bf Examples}
	
	\begin{itemize}
		\item[] $g$
		
		\item[] $(\lambda x. \ x \ x)$

		\item[] $((\lambda x. \ x \ x) \ g)$
	\end{itemize}

\end{frame}

\begin{frame}
	\frametitle{Notation Conventions}

	As with well-formed formulae in logic, we may opt to drop brackets. If you do, then all $\lambda$-expressions will be interpreted under the following conventions.

	\vspace{0.5cm}

	Application associates to the left e.g. 
	
	$t \ u \ v = (t \ u) \ v$

	\vspace{0.3cm}

	Abstraction associates to the right e.g. $\lambda x. \lambda y. \lambda z. \ t = \lambda x. (\lambda y. (\lambda z. \ t))$

	\vspace{0.3cm}

	Application binds tighter than abstraction e.g. $\lambda x. \ t \ u = \lambda x. \ (t \ u) \neq (\lambda x. \ t) u $ 

	\vspace{0.5cm}

	If you find brackets useful and clarifying, then feel free to keep them. 
	
\end{frame}

\begin{frame}
	\frametitle{Interpreting $\lambda$-expressions}

	All $\lambda$-expressions are to be thought of as functions.

	\vspace{0.5cm}

	Functions are first class citizens in the $\lambda$-calculus. 

	\vspace{0.5cm}

	Application, $(t \ u)$, is thought of as applying $t$ to the input $u$.

	\vspace{0.5cm}

	Abstraction $\lambda x. \ u$ is a function which takes input into $x$ and substitutes that value in every (free) instance of $x$ in the $\lambda$-expression $u$.

	\vspace{0.5cm}

	This means we have analogous free, bound, and substitution definitions as discussed for first-order logic and quantifiers. 

\end{frame}

\begin{frame}
	\frametitle{Free \& Bound Variables}

	The $\lambda$ abstraction operator bounds instances of its variable within the body of the abstraction expression. If $x$ is a variable and $e$ is a $\lambda$-expression, then $x$ is bound in any sub-expression of the form 
	
	$$\lambda x. \ e$$

	Any variable in $e$, other than $x$, is said to be free in the above sub-expression.

	\vspace{0.5cm}

	{\bf Examples}

	\begin{itemize}

		\item[] $(\lambda x.\lambda y. \ yx)$ \vspace{0.2cm}

		\item[] $(\lambda z. zx)(x) $ \vspace{0.2cm}

		\item[] $(\lambda x.\lambda y. \ yx)((\lambda z. zx)(x))$
	
	\end{itemize}
	
\end{frame}

\begin{frame}
	\frametitle{Substitution Rules}
	
	When we substitute one $\lambda$-expression $N$ into another $\lambda$-expression $M$ for a variable $x$ we replace all \emph{free occurences} of $x$ in $M$ with $N$: this is denoted $M[x:=N]$. 

	\vspace{0.2cm}

	Substitution of $\lambda$-expressions is defined inductively as follows

	\vspace{0.2cm}

	\begin{itemize}
		\item[] $x[x:=N] = N$ \vspace{0.2cm}
		\item[] $y[x:=N] = y$ when $y \neq x$ \vspace{0.2cm}
		\item[] $(M_{1} \ M_{2})[x:=N] = (M_{1}[x:=N] \ M_{2}[x:=N])$ \vspace{0.2cm}
		\item[] $(\lambda x. \ e)[x:=N] = \lambda x. \ e$ \vspace{0.2cm}
		\item[] $(\lambda y. \ e)[x:=N] = \lambda y. \ e[x:=N]$
	\end{itemize}

	As with substitution of well-formed formulae in first-order logic, this formal definition looks more complicated than the intuitive idea of substituting one term for another. 
	
\end{frame}

\section{Reduction Steps}

\begin{frame}
	\frametitle{$\alpha$-reduction}

	We do not want to make free variables bound when substituting.

	\vspace{0.2cm}

	Bound variables are \emph{dummy variables} in the sense that their name is not important. Compare the functions 

	$$f(x) = x^2 \hspace{2cm} f(t) = t^2$$

	The fact that one is written in terms of $x$, while the other is in terms of $t$ does not change the fact that these functions \emph{do} the same thing: square their input. 

	\vspace{0.5cm}

	We are free to rename bound variables in $\lambda$-expressions. This process is called $\alpha$-reduction. 

	\vspace{0.5cm}

	{\bf Examples}

	$\lambda x. \ x =_{\alpha} \lambda y. \ y$

	\vspace{0.1cm}

	$(\lambda x. \ (\lambda y. \ y x)) \ y =_\alpha$

\end{frame}

\begin{frame}
	\frametitle{$\beta$-reduction}

	This is intended to capture the idea that application $(M N)$ of one expression to another is to be thought of as applying the function $M$ to the input $N$.
	
	\vspace{0.5cm}

	Furthermore, the abstractions $\lambda x. \ e$ are intended to be interpreted as functions which take in an $x$ and substitute this into $e$. 

	\vspace{0.3cm}

	$$(\lambda x. \ e) \ M =_{\beta} e[x:=M]$$

	\vspace{0.3cm}

	$\lambda$-expressions of the form $\lambda x. \ e$ are called $\beta$-redexes.

	\vspace{0.3cm}

	Computation is: evaluating recursive functions, carrying out instructions for a Turing machine, and now computing $\beta$-reductions on $\lambda$-expressions.
	
\end{frame}

\section{Examples}
\begin{frame}
	\frametitle{Example}

	Perform a step of $\beta$-reduction on the following $\beta$-redex

	$$(\lambda x. \ x)(\lambda y. \ y (\lambda z. \ z w))$$
	% What does \lambda x. \ x  do here? Identity.

	\vspace{7cm}


\end{frame}

\begin{frame}
	\frametitle{Example}

	Perform a step of $\beta$-reduction on the following $\beta$-redex

	$$((\lambda x. \ \lambda y. \ y x) \ f) \ g $$
	% What does (\lambda x. \ \lambda y. y x)  do here? Prepares first function to be applied to the next. 

	\vspace{7cm}


\end{frame}


\begin{frame}
	\frametitle{Example}

	Perform a step of $\beta$-reduction on the following $\beta$-redex

	$$(\lambda x. \ x x x)(\lambda x. \ x x x)$$

	\vspace{7cm}


\end{frame}

\section{Currying}
\begin{frame}
	\frametitle{Multivariable Functions?}

	Multivariable functions are abundant in mathematics. Construction rules for $\lambda$-expressions only allow for the construction of unary functions. What gives? Currying means this actually isn't a problem.

	$$f(x,y) = x^{2}y + y^{2}x$$

	\vspace{7cm}

\end{frame}

\begin{frame}
	\frametitle{Further Reading}
	
	Here are some recommended reading to follow up on the lecture content. Some are freely available online. 

	\vspace{0.4cm}

	Type Theory and Functional Programming, \emph{Simon Thompson}
	
	\vspace{0.1cm}
	
	\begin{itemize}
		\item Stanford Encyclopedia Articles: 
			\begin{enumerate}
				\item The Lambda Calculus
				\item Type Theory
				\item Church's Type Theory
			\end{enumerate}
	\end{itemize}

	Church's original papers are worth visiting, although more work than Turing's paper. 

	An Unsolvable Problem of Elementary Number Theory, \emph{Alonso Church}

	\vspace{0.2cm}

	A Note on the Entscheidungsproblem, \emph{Alonso Church}
	
\end{frame}





\end{document}
