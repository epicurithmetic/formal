% Load required themes and packages.
\documentclass{beamer}
\usetheme{Pittsburgh}
\usecolortheme{default}
\useinnertheme{default}
\useoutertheme{default}
\usefonttheme{structurebold}
% Math related packges.
\usepackage{amsmath}
\usepackage{cancel}
\usepackage{soul}

% Natural Deduction package
\usepackage{proof}

% Package for hyperlinks
\usepackage{hyperref}

% Remove navigation bar
\beamertemplatenavigationsymbolsempty

% Bibliography
% \usepackage[style=verbose]{biblatex}
% \bibliography{deltaBib.bib}
% Command line sequence:
% pdflatex *filename*
% biber *filename*
% pdflatex *filename*

\title{Lecture 27: $\lambda$-Encoding Propositional Logic}
\author{MATH230}
\institute{Te Kura P\=angarau $\vert$ School of Mathematics and Statistics \\ Te Whare W\=ananga o Waitaha $\vert$ University of Canterbury}
\date{}

% Document body starts here.
\begin{document}


% Title frame
\begin{frame}

  \titlepage

\end{frame}

% Table of contents page
\begin{frame}
  \frametitle{Outline}

  \tableofcontents

\end{frame}

\section{Boolean $\lambda$-Expressions}

\begin{frame}
	\frametitle{Summary}

	In the previous lecture, motivated by conditional branching, we defined the following $\lambda$-expressions to mimic the Boolean values TRUE and FALSE: 

	\vspace{0.5cm}

	\begin{itemize}
		\item[] $\text{COND} = \lambda f. \ \lambda g. \ \lambda c. \ ((c \ f) \ g)$
		\item[] $\text{TRUE} = \lambda x. \ \lambda y. \ x$
		\item[] $\text{FALSE} = \lambda x. \ \lambda y. \ y$
		\item[] $\text{NOT} = \text{COND} \ \text{FALSE} \ \text{TRUE}$
	\end{itemize}

	\vspace{0.5cm}

	Where COND is a $\lambda$-expression which reduces to $f$ if $c = TRUE$ and reduces to $g$ if $c = FALSE$. 

	\vspace{0.3cm}

	You may expand and $\beta$-reduce the definition of NOT using the other three definitions above. However, you should perform any appropriate $\alpha$-reduction to make sure there are no variable name clashes.

\end{frame}

\begin{frame}
	\frametitle{Examples}

	When computing $\beta$-reduction on Booleans these redex will arise regularly. So lets simplify them now.

	\vspace{0.5cm}

	\begin{itemize}
		\item[] TRUE TRUE \vspace{1cm}
		\item[] TRUE FALSE \vspace{1cm}
		\item[] FALSE TRUE \vspace{1cm}
		\item[] FALSE FALSE \vspace{1cm} 
	\end{itemize}
	
\end{frame}

\section{Propositional Connectives}

\begin{frame}
	\frametitle{Propositional Connectives}

	In this lecture we will think about how to define the following propositional connectives and binary/Boolean functions:

	\vspace{0.3cm}

	\begin{itemize}
		%\item[] NOT 
		\item[] AND 
		\item[] OR 
		\item[] IMP(LICATION)
		\item[] NAND 
		\item[] EQUIV(ALENT) 
		\item[] IsTRUE?
		\item[] IsFALSE?
	\end{itemize}

	\vspace{3cm}

\end{frame}

\begin{frame}
	\frametitle{AND}

	AND is a binary function that returns a Boolean

	\begin{center}
		$\begin{array}{ c c | c }			
			P & Q & P \land Q\\
			\cline{1 - 3}
			T & T & T \\ 
			T & F & F \\ 
			F & T & F \\ 
			F & F & F
		\end{array}$
	\end{center}

	$\text{TRUE} = \lambda x. \ \lambda y. \ x$ \hspace{4cm} $\text{FALSE} = \lambda x. \ \lambda y. \ y$

	\vspace{0.5cm}

	$\text{AND} = \lambda p. \ \lambda q. \  $

	\vspace{5cm}
	
\end{frame}

\begin{frame}
	\frametitle{Example}

	$\beta$-reduce the following $\lambda$-expression to normal form.

	\vspace{0.5cm}

	$\text{AND} \ \text{TRUE} \ \text{FALSE}$

	\vspace{6cm}


\end{frame}

\begin{frame}
	\frametitle{OR}

	OR is a binary function that returns a Boolean

	\begin{center}
		$\begin{array}{ c c | c }			
			P & Q & P \lor Q\\
			\cline{1 - 3}
			T & T & T \\ 
			T & F & T \\ 
			F & T & T \\ 
			F & F & F
		\end{array}$
	\end{center}

	$\text{TRUE} = \lambda x. \ \lambda y. \ x$ \hspace{4cm} $\text{FALSE} = \lambda x. \ \lambda y. \ y$

	\vspace{0.5cm}

	$\text{OR} = \lambda p. \ \lambda q. \  $

	\vspace{5cm}
	
\end{frame}

\begin{frame}
	\frametitle{Example}

	$\beta$-reduce the following $\lambda$-expression to normal form.

	\vspace{0.5cm}

	$\text{OR} \ \text{FALSE} \ \text{TRUE}$

	\vspace{6cm}


\end{frame}

\begin{frame}
	\frametitle{IMP}

	IMPlication is a binary function that returns a Boolean

	\begin{center}
		$\begin{array}{ c c | c }			
			P & Q & P \to Q\\
			\cline{1 - 3}
			T & T & T \\ 
			T & F & F \\ 
			F & T & T \\ 
			F & F & T
		\end{array}$
	\end{center}

	$\text{TRUE} = \lambda x. \ \lambda y. \ x$ \hspace{4cm} $\text{FALSE} = \lambda x. \ \lambda y. \ y$

	\vspace{0.5cm}

	$\text{IMP} = \lambda p. \ \lambda q. \  $

	\vspace{5cm}
	
\end{frame}

\begin{frame}
	\frametitle{Example}

	$\beta$-reduce the following $\lambda$-expression to normal form.

	\vspace{0.5cm}

	$\text{IMP} \ \text{TRUE} \ \text{FALSE}$

	\vspace{6cm}


\end{frame}

\begin{frame}
	\frametitle{NAND}

	NAND is a binary function that returns a Boolean

	\begin{center}
		$\begin{array}{ c c | c }			
			P & Q & \text{NAND}PQ\\
			\cline{1 - 3}
			T & T & F \\ 
			T & F & T \\ 
			F & T & T \\ 
			F & F & T
		\end{array}$
	\end{center}

	$\text{TRUE} = \lambda x. \ \lambda y. \ x$ \hspace{4cm} $\text{FALSE} = \lambda x. \ \lambda y. \ y$

	\vspace{0.5cm}

	$\text{NAND} = \lambda p. \ \lambda q. \  $

	\vspace{5cm}
	
\end{frame}

\section{Boolean Equivalence}

\begin{frame}
	\frametitle{EQUIV}

	EQUIV is a binary function that returns a Boolean

	\begin{center}
		$\begin{array}{ c c | c }			
			P & Q & \text{EQUIV}PQ\\
			\cline{1 - 3}
			T & T & T \\ 
			T & F & F \\ 
			F & T & F \\ 
			F & F & T
		\end{array}$
	\end{center}

	% This one is tricky. One way to get it is to find a propositional formula involving the connectives that defines this function.

	% $=pq := (p \land q) \lor \lnot(p \lor q)$

	$\text{TRUE} = \lambda x. \ \lambda y. \ x$ \hspace{4cm} $\text{FALSE} = \lambda x. \ \lambda y. \ y$

	\vspace{0.5cm}

	$\text{EQUIV} = \lambda p. \ \lambda q. \  $

	\vspace{5cm}
	
\end{frame}
   
\begin{frame}
	\frametitle{Example}

	$\beta$-reduce the following $\lambda$-expression to normal form.

	\vspace{0.5cm}

	$\text{EQUIV} \ \text{FALSE} \ \text{FALSE}$

	\vspace{6cm}


\end{frame}

\begin{frame}
	\frametitle{IsTRUE?}

	IsTRUE? is a unary function that returns a Boolean.

	\begin{center}
		$\begin{array}{ l | c }			
			A &  \text{IsTRUE?} A\\
			\cline{1 - 2}
			T &  T \\ 
			F &  F
		\end{array}$
	\end{center}

	$\text{TRUE} = \lambda x. \ \lambda y. \ x$ \hspace{4cm} $\text{FALSE} = \lambda x. \ \lambda y. \ y$

	\vspace{0.5cm}

	$\text{IsTRUE?} = $

	\vspace{3cm}

\end{frame}

\begin{frame}
	\frametitle{IsFALSE?}

	IsFALSE? is a unary function that returns a Boolean.

	\begin{center}
		$\begin{array}{ l | c }			
			A &  \text{IsFALSE?} A\\
			\cline{1 - 2}
			T &  F \\ 
			F &  T
		\end{array}$
	\end{center}

	$\text{TRUE} = \lambda x. \ \lambda y. \ x$ \hspace{4cm} $\text{FALSE} = \lambda x. \ \lambda y. \ y$

	\vspace{0.5cm}

	$\text{IsFALSE?} = $

	\vspace{3cm}

\end{frame}

\section{Summary}

\begin{frame}
	\frametitle{Propositional Logic Summary}

	This is a list of encodings for propositional logic into $\lambda$-expressions. Note that there are many ways to encode these expressions.

	\vspace{0.5cm}

	\begin{itemize}
		\item[] $\text{COND} = \lambda f. \ \lambda g. \ \lambda c. \ ((c \ f) \ g)$
		\item[] $\text{TRUE} = \lambda x. \ \lambda y. \ x$
		\item[] $\text{FALSE} = \lambda x. \ \lambda y. \ y$
		\item[] $\text{NOT} = \text{COND} \ \text{FALSE} \ \text{TRUE}$
		\item[] $\text{AND} = \lambda p. \ \lambda q. \ (p \ q) \ p$
		\item[] $\text{OR} = \lambda p. \ \lambda q. \ (p \ p) \ q$
		\item[] $\text{IMP} = \lambda p. \ \lambda q. \ \text{OR} \ (\text{NOT} \ p) \ q$
		\item[] $\text{EQUIV} = \lambda p. \ \lambda q. \ (p \ q) (NOT \ q)$
		%\item[] $\text{EQUIV} = \lambda p. \ \lambda q. \ \text{OR} \ (\text{AND} \ p \ q) \ (\text{NOT} \ (\text{OR} \ p \ q))$
		\item[] $\text{IsTRUE?} = I$
		\item[] $\text{IsFALSE?} = \text{NOT}$
	\end{itemize}

\end{frame}

% \section{Glimpse Towards Curry-Howard Correspondence}
% \begin{frame}
% 	\frametitle{Logical v. $\beta$-Equivalence}
% 	% At least true in spirit: morally true. Something probably needs to be said about types in the lambda calculus. Maybe some other details required too. 

% 	Logical equivalences raise an interesting question.

% 	$$ P \lor Q \dashv \vdash \lnot(\lnot P \land \lnot Q) $$

% 	\vspace{0.2cm}

% 	Suppose we encode both sides of a logical equivalence as $\lambda$-expressions using the encoding we have developed above. 

% 	\vspace{0.2cm}

% 	\begin{itemize}
% 		\item[] Do the two functions agree on Booleans? Yes!
% 		\item[] Are the two $\lambda$-expressions $\beta$-equivalent? 
% 	\end{itemize}

% 	\vspace{0.2cm}

% 	The answer to the second question is Yes if and only if the equivalence can be witnessed with an intuitionitistic proof!

% 	\vspace{0.2cm}

% 	{\bf Curry-Howard Correspondence}

% 	There is a deep relationship between intuitionitistic proofs and $\beta$-reduction. Intuitionitistic proofs are precisely those with computational content!
	
% \end{frame}


\begin{frame}
	\frametitle{Further Reading}
	
	Here are some recommended reading to follow up on the lecture content. Some are freely available online. 

	\vspace{0.4cm}

	Type Theory and Functional Programming, \emph{Simon Thompson}
	
	\vspace{0.1cm}
	
	\begin{itemize}
		\item Stanford Encyclopedia Articles: 
			\begin{enumerate}
				\item The Lambda Calculus
				\item Type Theory
				\item Church's Type Theory
			\end{enumerate}
	\end{itemize}

	Church's original papers are worth visiting, although more work than Turing's paper. 

	An Unsolvable Problem of Elementary Number Theory, \emph{Alonso Church}

	\vspace{0.2cm}

	A Note on the Entscheidungsproblem, \emph{Alonso Church}
	
\end{frame}





\end{document}
