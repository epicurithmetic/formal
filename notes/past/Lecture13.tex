% Load required themes and packages.
\documentclass{beamer}
\usetheme{Pittsburgh}
\useinnertheme[shadow]{rounded}
\setbeamercolor{block title}{bg=blue!30, fg=black}
\setbeamercolor{block body}{bg=blue!20}
\usecolortheme{default}
\useinnertheme{default}
\useoutertheme{default}
\usefonttheme{structurebold}
% Math related packges.
\usepackage{amsmath}
\usepackage{cancel}

% Natural Deduction package
\usepackage{proof}

% Package for hyperlinks
\usepackage{hyperref}

% Remove navigation bar
\beamertemplatenavigationsymbolsempty

% Bibliography
% \usepackage[style=verbose]{biblatex}
% \bibliography{deltaBib.bib}
% Command line sequence:
% pdflatex *filename*
% biber *filename*
% pdflatex *filename*

\newtheoremstyle{indentDefn}
{\topsep} % Space above
{\topsep} % Space below
{\it} % Body font
{2cm} % Indent amount
{\bf} % Theorem head font
{:} % Punctuation after theorem head
{0.5em} % Space after theorem head
{} % Theorem head spec

\theoremstyle{indentDefn} \newtheorem{defn}[]{Definition}

\title{Lecture 13: Proof Theory II}
\subtitle{Syntax of the Existential Quantifier}
\author{MATH230}
\institute{Te Kura P\=angarau $\vert$ School of Mathematics and Statistics \\ Te Whare W\=ananga o Waitaha $\vert$ University of Canterbury}
\date{}

% Document body starts here.
\begin{document}


% Title frame
\begin{frame}
	
	\titlepage
	
\end{frame}

% Table of contents page
\begin{frame}
	\frametitle{Outline}
	
	\tableofcontents
	
\end{frame}



\section{Rules of Inference: Existential Quantifier}

\begin{frame}
	\frametitle{Existential Introduction}
	
	If $^{\Sigma}_{\alpha[x/y]}\mathcal{D}$ is a derivation, where $y$ is free for $x$ in $\alpha$, then
	
	\vspace{0.5cm}
	
	\begin{center}
		$\begin{array}{c}
		\infer[\exists I]{\exists x \alpha}{\begin{array}{c} 
			\Sigma \\
			\mathcal{D} \\ 
			\alpha[x/y]			
			\end{array}}
		\end{array}$
	\end{center}
	
	is a derivation of $\exists x \alpha$ from hypotheses $\Sigma$. 
	
\end{frame}

\begin{frame}
	\frametitle{Example}
	
	Show $\forall x Px \ \vdash \ \exists Px$
	
	\vspace{6cm}
	
	
\end{frame}

\begin{frame}
	\frametitle{Example}
	
	Show $\{\forall x (Px \rightarrow Qx), \lnot Qa\} \ \vdash \ \exists x \ \lnot Px$
	
	\vspace{6cm}
	
	
\end{frame}

\begin{frame}
	\frametitle{Existential Elimination}
	
	If $^{\Sigma_{1}}_{\exists x \alpha}\mathcal{D}_{1}$, $^{\Sigma_{2}}_{\alpha[x/y]\rightarrow\gamma}\mathcal{D}_{2}$, and $y$ is neither free in $\gamma$ nor $\Sigma$, then
	
	\begin{center}
	$\begin{array}{ c }
	
	\infer[\exists E]{\gamma}
	{
		\begin{array}{c} \Sigma_{1}  \\ \mathcal{D}_{1} \\ \exists x \alpha \end{array}
		& 
		\begin{array}{c} \Sigma_{2}  \\ \mathcal{D}_{2} \\ \alpha[x/y]\rightarrow\gamma \end{array}				
	}	
	
	\end{array}$
	\end{center}

	is a derivation of $\gamma$ from hypotheses $\Sigma_{1} \cup \Sigma_{2}$. 
	
	% Take time to explain that if $y$ is arbitary, then we can put in the term we know to exist from the first deduction. The fact that $y$ is arbitrary is encoded by not being free in $\gamma$ nor $\Sigma$. 
	
\end{frame}

\section{Examples}

\begin{frame}
	\frametitle{Example}
	
	Show $\{\forall x (Px \rightarrow Qx), \ \exists x Px\} \ \vdash \ \exists x Qx$
	
	% There are a few wrong things one might do here which are worth pointing out. 
	
	\vspace{6cm}
	
\end{frame}

\begin{frame}
	\frametitle{Example: When Things Go Wrong I}
	
	Sherlock is a detective. Therefore, everyone is a detective. 
	
	\vspace{0.5cm}
	
	$$ \exists x Dx \ \vdash \ \forall x Dx $$
	
	\vspace{0.5cm}
	
	\begin{center}
		$\begin{array}{c}
			\infer[\forall I]{\forall x Dx}{\infer[\exists E]{Da}
					{\exists x Dx & \infer[\rightarrow I]{Da \rightarrow Da}{\infer[]{ \ Da \ }{}}
				}
			}
		
		 \end{array}$
	\end{center}

	
\end{frame}
	
\begin{frame}
	\frametitle{Example: When Things Go Wrong II}
	
	Sherlock is a human. Gladstone is a dog. Therefore, there exists somthing that is both a human and a dog. 
	
	\vspace{0.5cm}
	
	$$ Hs, \ \exists x Dx \ \vdash \ \exists x (Hx \land Dx) $$
	
	\vspace{0.5cm}
	
	\begin{center}
		$\begin{array}{c}
	
			\infer[\exists E]{\exists x (Hx \land Dx)}{\exists x Dx & \infer[\rightarrow I]{Ds \rightarrow \exists x (Hx \land Dx)}
				{\infer[\exists I]{\exists x (Hx \land Dx)}
					{\infer[\land I]{Hs \land Ds}
						{Hs & \infer[]{Ds}
							{}}}}}


	     \end{array}$
	\end{center}
\end{frame}

\section{Extended Example}

\begin{frame}
	\frametitle{Proof Strategy}
	
	\begin{center}
	Show $\exists x Px \lor \exists x Qx \ \vdash \ \exists x (Px \lor Qx)$
	\end{center}

	{\bf Note:} Our hypothesis is a disjunction. So we will need the disjunction elimination rule to make use of it. 
	
	\vspace{2cm}
	
	This breaks the proof into two steps: 
	
	\begin{center}
	\begin{enumerate}
		\item $\exists x Px \rightarrow \exists x (Px \lor Qx)$
		\item $\exists x Qx \rightarrow \exists x (Px \lor Qx)$		
	\end{enumerate}
	\end{center}
	
	Both of these steps will have the same form, so let's focus on the first.
		
\end{frame}

\begin{frame}
	\frametitle{Proof Strategy}

	We have reduced the problem to showing:	

	 $$ \ \vdash \ \exists x Px \rightarrow \exists x (Px \lor Qx)$$
	
	\pause
	
	The deduction theorem ($\rightarrow I$) implies this is equivalent to showing:
	
	$$\exists x Px \ \vdash \  \exists x (Px \lor Qx)$$
	
	\pause
	
	Now our hypothesis has an existential quantifier in it, so we need to use the corresponding elimination rule. Our proof will have a line of the form: 
	
	\vspace{0.7cm}
	
	$$ \infer[\exists E]{\exists x (Px \lor Qx)}{\exists x Px \ & \infer[?]{\ Pa \rightarrow \exists x (Px \lor Qx)}{} } $$

	How can we complete the proof? 
		
\end{frame}

\begin{frame}
	\frametitle{Proof Strategy}
	
		We want to show $ \ \vdash \ \exists x Px \rightarrow \exists x (Px \lor Qx)$ and so far our proof looks like this. What can we do to finish this proof?
	
	$$ \infer[\exists E]{\exists x (Px \lor Qx)}{\infer[]{\exists x Px}{} \ & \infer[\rightarrow I]{\ Pa \rightarrow \exists x (Px \lor Qx)}{\infer[\exists I]{\exists x (Px \lor Qx)}
			{\infer[\lor I]{Pa \lor Qa}
				{\infer[]{ \ Pa \ }{}}}}} $$
	
	\vspace{3cm}		
	
\end{frame}

\begin{frame}
	\frametitle{Proof Strategy}
	
	We have proven the first of our subgoals. Go through the same process to prove the second subgoal. 
	
	\begin{center}
		\begin{enumerate}
			\item $\exists x Px \rightarrow \exists x (Px \lor Qx)$
			\item $\exists x Qx \rightarrow \exists x (Px \lor Qx)$		
		\end{enumerate}
	\end{center}	
	
	Use the disjunction elimination to tie these goals together into a proof of the original statement
	
	$$\exists x Px \lor \exists x Qx \ \vdash \ \exists x (Px \lor Qx)$$
	
	\vspace{2cm}
	
	%\hspace{6cm}\footnotesize{See tutorials for a full proof.}
	
\end{frame}

\section{G\"{o}del's Completeness Theorem}

\begin{frame}
	\frametitle{Metalogical Theorems}
	
	G\"{o}del proved that first-order predicate logic is both sound and complete. 
	
	$$ \Sigma \vDash \alpha \quad \text{if and only if} \quad \Sigma \vdash \alpha$$
	
	Thus we can provide deductions for all, and only those, wff that are true in every model satisfying the hypotheses $\Sigma$. 
	
	\vspace{0.5cm}
	
	We will not prove this. But it's worth noting that the proofs follow the same form as the proofs for the soundness and completeness of propositional logic. 
	
\end{frame}


\begin{frame}
  \frametitle{Further Reading}


Below is a selection of resources that I used to prepare the lecture. You might like to read over them yourself to help get a more complete picture of the topics discussed. 

\vspace{0.5cm}

\begin{itemize}
	\item L$\exists \forall$N, \emph{Logic and Proof}. Sections:  7 and 8.
	\item Logic Matters, \emph{An Introduction to Formal Logic}: Not so helpful for our version of deduction. But you might like to read Chapter 31, 32, and 33. 
\end{itemize}


\end{frame}





\end{document}
