% Load required themes and packages.
\documentclass{beamer}
\usetheme{Pittsburgh}
\useinnertheme[shadow]{rounded}
\setbeamercolor{block title}{bg=blue!30, fg=black}
\setbeamercolor{block body}{bg=blue!20}
\usecolortheme{default}
\useinnertheme{default}
\useoutertheme{default}
\usefonttheme{structurebold}
% Math related packges.
\usepackage{amsmath}
\usepackage{cancel}

% Natural Deduction package
\usepackage{proof}
% Bibliography
% \usepackage[style=verbose]{biblatex}
% \bibliography{deltaBib.bib}
% Command line sequence:
% pdflatex *filename*
% biber *filename*
% pdflatex *filename*

% Remove navigation bar
\beamertemplatenavigationsymbolsempty

\newtheoremstyle{indentDefn}
{\topsep} % Space above
{\topsep} % Space below
{\it} % Body font
{2cm} % Indent amount
{\bf} % Theorem head font
{:} % Punctuation after theorem head
{0.5em} % Space after theorem head
{} % Theorem head spec

\theoremstyle{indentDefn} \newtheorem{defn}[]{Definition}


\title{Lecture 2: Propositional Logic}
\author{MATH230}
\institute{Te Kura P\=angarau $\vert$ School of Mathematics and Statistics \\ Te Whare W\=ananga o Waitaha $\vert$ University of Canterbury}
\date{}

% Document body starts here.
\begin{document}


% Title frame
\begin{frame}

  \titlepage

\end{frame}

% Table of contents page
\begin{frame}
  \frametitle{Outline}

  \tableofcontents

\end{frame}

% As we discussed yesterday, we require a language in which we can express all of mathematics so that we can better trust the work that is done and continues to be done. 

% Logic is the language we will use. So, we will start here with logic. 

\section{Arguments}

\begin{frame}
  \frametitle{Arguments and Proofs}

Analysis of a proof in mathematics, or an argument in a natural language, relies on how particular statements are connected to each other. How one or more statements are said to entail others. 

\vspace{1cm}

 {\bf Example: }
 
 If $p$ divides $ab$, then $p$ divides $a$ or $p$ divides $b$.\footnote{This is an example of an argument: this statement is not necessarily true!}
  
 \vspace{1cm}
 
 It is the connective words, as much as the mathematical content words, that we have to analyse when deciding whether this is a good argument.



\end{frame}

\begin{frame}
	\frametitle{Example: Natural Language}
	
	\begin{itemize}
		\item If Watson moves in with Holmes, then Holmes will be forever annoyed. Watson moved in with Holmes. Therefore Holmes will be forever annoyed.
		\vspace{1cm}
		\item If Watson can trap Moriarty, then Holmes can. Holmes can't trap Moriarty. Therefore Watson can't.
		\vspace{1cm}
		\item Either Holmes catches Moriarty or the world will fall into chaos. The world has fallen into chaos. Therefore Holmes did not catch Moriarty.
		
	\end{itemize}
	
	
\end{frame}

% Each of these examples are built up of declaritive statements (propositions), one of which is singled out (by the therefore)  as a conclusion. The rest we call premises. 

\begin{frame}
	\frametitle{Argument}
	

	\begin{defn}
		An argument is a set of declaritive sentences (propositions), one of which is singled out as the conclusion, while the rest are considered premises. 
	\end{defn}
		 
	\vspace{0.5cm}
	
	Premises are the evidence to support the conclusion.
	

\end{frame}

\begin{frame}
	\frametitle{Example: Natural Language}
	
	\begin{itemize}
		\item If Watson moves in with Holmes, then Holmes will be forever annoyed. Watson moved in with Holmes. Therefore, Holmes will be forever annoyed.

		\vspace{0.5cm}

		Let's break this up into premises and conclusion: 
		
		\vspace{5cm}
		
		% P1: If Watson moves in with Holmes, then Holmes will be forever annoyed. 
		% P2: Watson moved in with Holmes. 
		%  C: Holmes will be forever annoyed.
		
		% Ask questions about the structure of P1 and P2.
	\end{itemize}
\end{frame}

\begin{frame}
	\frametitle{Propositional Structures}
	
	\begin{defn} 
		
		An atomic proposition has no propositional substructure.
		
	\end{defn}


	
	We saw above that some propositions do have extra structure: ``If... , then...." and ``Either .... or ... " and ``can't" are important to the nature of the argument. 
	
	\vspace{0.5cm}
	
	Such connectives are used to join atomic propositions into compound propositions.
	
	
	
\end{frame}

\begin{frame}
	\frametitle{Example: Natural Language}
	
	\begin{itemize}
		\item Either Holmes catches Moriarty or the world will fall into chaos. The world has fallen into chaos. Therefore Holmes did not catch Moriarty.
		
		\vspace{0.5cm}
		
		Let's break this up into premises and conclusion and determine the atomic propositions.
		
		\vspace{5cm}
		
		% P1: If Watson moves in with Holmes, then Holmes will be forever annoyed. 
		% P2: Watson moved in with Holmes. 
		%  C: Holmes will be forever annoyed.
		
		% Ask questions about the structure of P1 and P2.
	\end{itemize}
\end{frame}

\begin{frame}
	\frametitle{Comment on Natural Language}
	
	% Recall the example about numbers.
	% Recall the first lecture about the need to choose axioms. 
	
	It can be subjective as to what one considers atomic. Natural language is very rich and open to many interpretations. As we are hoping for a precise language on which to found mathematics, we need to remove these ambiguities. 
	
	\vspace{1cm}
	
	For this reason we introduce the following notations, our first formal logic, collectively known as (classical) propositional logic.
	
	
\end{frame}

\section{Propositional Logic}

\begin{frame}
	\frametitle{Binary Connectives}
	
	To express the same syntactic structure of an \emph{argument} without the ambiguities of a natural language we use capital (English) letters to denote atomic propositions, called \emph{propositional variables}. We use the following symbols to express compound propositions: 
	
	\vspace{0.5cm}
	
	\begin{itemize}
		\item $\lnot$ : ``It is not the case that... " or "Not... "
		\item $\land$ : ``Both... and ... " 
		\item $\lor$ : ``Either... or ... "
		\item $\rightarrow$ : ``If... , then ... "
		\item $\leftrightarrow$ : `` ... if and only if ... "
	\end{itemize}
	
\end{frame}

\begin{frame}
	\frametitle{Example: Translating from NL to PL}
	\begin{itemize}
		\item If Watson can trap Moriarty, then Holmes can. Holmes Can't trap Moriarty. Therefore Watson can't.
	\end{itemize}
	
	\vspace{5cm}
	
	
	
	
	
	
\end{frame}	

\begin{frame}
	\frametitle{Grammar}
	
	Our language is further made up of \emph{well-formed formulae} which we define inductively as follows:	
	
	\vspace{0.5cm}
	\begin{defn}[Well-Formed Formulae]
	\begin{itemize}
		\item {\bf Atomic Formulae:} If $\alpha$ is a single propositional variable, then $\alpha$ is a wff.
		\item {\bf Negation:} If $\alpha$ is a wff, then $\lnot\alpha$ is a wff. 
		\item {\bf Binary Connective:} If $\alpha$ and $\beta$ are wff and $*$ is a binary connective, then $(\alpha * \beta)$ is a wff. 		
	\end{itemize}
	\end{defn}
	
	\vspace{0.5cm}	
	
	{\bf Note:} Any expression that is not a wff by virtue of these three constructions is not a wff. 
	
\end{frame}

\begin{frame}
	\frametitle{Examples}
	Which of the following are wff in propositional logic?
	
	\begin{itemize}
		\item[1.] $A$
		\item[2.] $AB$
		\item[3.] $(A \rightarrow B)$
		\item[4.] $A \rightarrow B \rightarrow C$
		\item[5.] $((A \rightarrow B) \rightarrow C)$
		\item[6.] $\lnot Q$
		\item[7.] $A \lor Q$
		\item[8.] $A \rightarrow \lnot B \lor C$		
	\end{itemize}
		
	\pause
		
	{\bf Binding Conventions:} %If in doubt, then stick to using parentheses!
	\begin{itemize}
		\item $\lnot$ binds most tightly, 
		\item $\lor$ and $\land$ bind more tightly than $\rightarrow$,
		\item $\rightarrow$ binds more tightly than $\leftrightarrow$. 			
	\end{itemize}
	
\end{frame}



\begin{frame}
  \frametitle{Further Reading}

  Here are some recommended reading to follow up on the lecture content. They are all freely available online. 

	\vspace{0.5cm}

	\begin{itemize}
		\item L$\exists \forall$N, \emph{Logic and Proof}. Sections: 3.1, 6.1, and 6.2 
		\item Logic Matters, \emph{An Introduction to Formal Logic}: p172 - 216
	\end{itemize}

\end{frame}





\end{document}
