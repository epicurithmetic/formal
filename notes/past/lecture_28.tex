% Load required themes and packages.
\documentclass{beamer}
\usetheme{Pittsburgh}
\usecolortheme{default}
\useinnertheme{default}
\useoutertheme{default}
\usefonttheme{structurebold}
% Math related packges.
\usepackage{amsmath}
\usepackage{cancel}
\usepackage{soul}

% Natural Deduction package
\usepackage{proof}

% Package for hyperlinks
\usepackage{hyperref}

% Remove navigation bar
\beamertemplatenavigationsymbolsempty

% Bibliography
% \usepackage[style=verbose]{biblatex}
% \bibliography{deltaBib.bib}
% Command line sequence:
% pdflatex *filename*
% biber *filename*
% pdflatex *filename*

\title{Lecture 28: Church Numerals}
\subtitle{Related Data Structures}
\author{MATH230}
\institute{Te Kura P\=angarau $\vert$ School of Mathematics and Statistics \\ Te Whare W\=ananga o Waitaha $\vert$ University of Canterbury}
\date{}

% Document body starts here.
\begin{document}


% Title frame
\begin{frame}

  \titlepage

\end{frame}

% Table of contents page
\begin{frame}
  \frametitle{Outline}

  \tableofcontents

\end{frame}

\section{Church Numerals}

\begin{frame}{Church Numerals}

    All that is available to us is application and abstraction. However, that is enough to encode the behaviour of the natural numbers in the $\lambda$-calculus. 
    
    \begin{align*}
        \text{ZERO} &= \lambda s. \ \lambda x. \ x \\
        \text{ONE} &= \lambda s. \ \lambda x. \ s(x) \\
        \text{TWO} &= \lambda s. \ \lambda x. \ s(s(x)) \\
        \text{THREE} &= \lambda s. \ \lambda x. \ s(s(s(x))) \\
        &\vdots \\
        n & = \lambda s. \ \lambda x. \ s(s \dots (s(x)) \dots ) \\
        &\vdots
    \end{align*}
        
    The Church numeral representing the natural number $M$ is a binary $\lambda$-expression that applies the first argument to the second $M$ times.    

\end{frame}

\begin{frame}{Pattern}

    If $n$ is a Church numeral and $f,a$ are arbitrary function symbols, then we have the following patterns that occur often in computations with Church numerals. 
    
    \vspace{0.3cm}
    
    The $\lambda$-expression $nf$ is a function which applies $f$ $n$ times to its input. That is, $f$ composed with itself $n$ times. 
    
    \vspace{0.3cm}
    
    So we can read $$nfa$$ as \emph{$``f"$ applied to $``a"$ repeatedly $``n"$ times}.
    
    \vspace{0.3cm}
    
    If we follow this idea when $f$ is also a Church numeral, then we get a $\lambda$-expression for exponentiation.
    
\end{frame}

\begin{frame}{Observation}
    % Explain the beta reduction pattern of one church numeral applied to another. 
    \begin{align*}
        \text{TWO} \ \text{ONE} &= (\lambda u. \ \lambda v. \ u(u(v))) (\lambda s. \ \lambda x. \ s(x)) \\
        &= \lambda v. \ (\lambda s. \ \lambda x. \ s(x))((\lambda s. \ \lambda x. \ s(x))(v))) \\
        &= \lambda v. \ (\lambda s. \ \lambda x. \ s(x))(\lambda x. \ v(x)) \\
        &= \lambda v. \ (\lambda s. \ \lambda x. \ s(x))(\lambda w. \ v(w)) \\
        &= \lambda v. \ (\lambda x. \ (\lambda w. \ v(w))(x)) \\
        &= \lambda v. \ \lambda x. \ v(x) \\
        &= \text{ONE}
    \end{align*}
    
    \begin{align*}
        \text{ONE} \ \text{TWO} &= (\lambda s. \ \lambda x. \ s(x))(\lambda u. \ \lambda v. \ u(u(v))) \\
        &= (\lambda x. \ (\lambda u. \ \lambda v. \ u(u(v)))(x)) \\
        &= (\lambda x. \ \lambda v. \ x(x(v)))) \\
        &= \text{TWO}
    \end{align*}
\end{frame}

\begin{frame}{Example: Exponential}
    We abstract over this idea to define a $\lambda$-expression to compute exponentiation.
    
    $$\text{EXP} = \lambda e. \ \lambda b. \ eb $$
    
    {\bf Example} 
    
    $\beta$-reduce the following expression to normal form
    
    \vspace{0.3cm}
    
    $\text{EXP} \ \text{THREE} \ \text{TWO}$
    \vspace{3cm}
\end{frame}

\section{Church Arithmetic}

\begin{frame}{Programming in $\lambda$-Calculus}

    % Functions in the $\lambda$-calculus (or functional programming languages) are often said to ``not have names". 
    
    % \vspace{0.5cm}
    
    When programming and running computations in this language we do not update named spaces in memory. 
    
    \vspace{0.5cm}
    
    We can't think about updating a number stored in a named variable. There is no syntax for this updating in the $\lambda$-calculus.
    
    \vspace{0.5cm}
    
    Each time we calculate a new $\lambda$-expression (e.g. Church numeral) we must construct it, from scratch, using the input numerals. 
    
\end{frame}

\begin{frame}{Encoding Arithmetic Functions}

    We will now find $\lambda$-expressions for fundamental arithmetic functions and predicates on Church numerals.
    
    {\bf Arithmetic Functions}
    \begin{itemize}
        \item[] SUCC
        \item[] SUM
        \item[] MULT
        \item[] EXP
        \item[] PRED
        \item[] SUB
    \end{itemize}
    
    {\bf Arithmetic Predicates}
    \begin{itemize}
        \item[] ZERO?
        \item[] GREATER?
        \item[] EQUAL?
    \end{itemize}
    
\end{frame}

\begin{frame}{Encoding Arithmetic Functions}

    % This is not a state based model of computation. 
    
    % \vspace{0.2cm}
    
    % There are no numbers in memory that we update with a program. 
    
    % \vspace{0.2cm}
    
    Programs in the $\lambda$-calculus need to {\bf construct} the output.
    
    \vspace{0.3cm}
    
    Unary functions on Church numerals will always start 
    
    $$\lambda n. \ \lambda u. \ \lambda v. \ \langle \text{BODY} \rangle$$
    
    Binary functions on Church numerals will always start 
    
    $$\lambda m. \ \lambda n. \ \lambda u. \ \lambda v. \ \langle \text{BODY} \rangle $$
    
    \vspace{0.3cm}
    
    The first abstractions are for the inputs to the function. 
    
    \vspace{0.3cm}
    
    Second abstractions ($u$,$v$) are to construct the output numeral. 

\end{frame}


\begin{frame}{Successor}

    The successor is a unary function that returns a numeral with one more function application of the first argument to the second.
    
    \vspace{0.3cm}
    
    $\text{SUCC} = \lambda n. \ \lambda u. \ \lambda v. \ $
    
    \vspace{6cm}
        
\end{frame}

\begin{frame}{Example}
    SUCC ZERO
    
    \vspace{7cm}  
\end{frame}

\begin{frame}{SUM}

    The sum of two Church numerals $m,n$ is a binary function that returns a numeral with $m+n$ applications of the first argument to the second. This is similar to string concatenation of successors.
    
    \vspace{0.3cm}
    
    $\text{SUM} = \lambda m. \ \lambda n. \ \lambda u. \ \lambda v. \ $
    
    \vspace{6cm}

\end{frame}

\begin{frame}{Example}
    SUM ONE ONE
    
    \vspace{7cm} 
\end{frame}

\begin{frame}{MULT}
    If $m,n$ are Church numerals, then the output of multiplication requires $n$ applications $m$ times of the first argument to the second. 
    
    \vspace{0.3cm}
    
    $\text{MULT} = \lambda m. \ \lambda n. \ \lambda u. \ \lambda v. \ $
    
    \vspace{6cm}      
\end{frame}

\begin{frame}{Example}

    MULT TWO TWO
    
    \vspace{7cm}   
    
\end{frame}

\begin{frame}{ZERO?}
    % 0 = FALSE with this encoding. 
    
    Given Church numeral $m$ how do we test if it is ZERO?
    
    \vspace{0.3cm}
    
    $\text{ZERO?} = \lambda m.  \ $
    
    \vspace{6cm}  
    
\end{frame}

\begin{frame}{Predecessor}

    % Apparently Church once thought this could not be done in the lambda calculus 
    % (0,0) -> (1,0) -> (2,1) -> (3,2) -> ...
    % This can all be encoded in the lambda calculus
    
    To one way of thinking, we need to *remove* one application of the function in the Church numeral. 
    
    \vspace{1.5cm}
    
    However that way of thinking is ``state based" - as if we have an object somewhere in some memory and we update its properties. 
    
    \vspace{0.5cm}
    
    This is not the way programming is done in the $\lambda$-calculus. 
    
    \vspace{0.5cm}
    
    Instead we need to think, given an input Church numeral $n$ how do we construct the Church numeral representing $n-1$?
    
    \vspace{2cm}
    
\end{frame}

\section{Pairs and Lists}

\begin{frame}{PAIR}

    We have been treating applications of the form $ab$ as if they were pairs. Let us formalise this idea with a function to CONStruct a pair from two inputs.
    
    $$ \text{PAIR} = \lambda x. \lambda y. \lambda f. \ f \ x \ y$$
    
    Once a pair is constructed, we may use the following methods to retrieve either the first or second element respectively. 
    
    $$\text{FST} = \lambda u. \ \lambda v. \ u \hspace{1cm} \text{SCD} = \lambda u. \ \lambda v. \ v $$
    
    {\bf Example}
    
    PAIR ONE TWO FST  $=_{\beta}$ ONE
    
    PAIR ONE TWO SCD  $=_{\beta}$ TWO
    
    PAIR ONE (PAIR TWO THREE) SCD $=_{\beta}$ PAIR TWO THREE
    
\end{frame}

% \begin{frame}{Pairs of pairs of pairs ... LISTs?}

%     % Construct nested pairs and call them lists
%     % (a, (b, (c, (d,e))))
%     % Build by iterating the pair function
    
%     % Were lists invented by Kleene while defining the predecessor function?
    
%     We can use pairs of pairs of... to construct the LIST data structure in the $\lambda$-calculus. Or any functional programming language built on (or, modeled off of) the $\lambda$-calculus.
    
%     \vspace{0.5cm}
    
%     In order to design methods to effectively manipulate LISTs we need to talk about recursion in the $\lambda$-calculus. 
    
%     \vspace{0.5cm}
    
%     {\bf Exercise} Without using recursion determine a $\lambda$-expression for retrieving the $n$th element of a list. Assume the list has at least $n$ elements.
    
%     % $\lambda n. \lambda \ell. \ (\ell \ ((\text{PRED} \ n) \ \text{SCD})) \ \text{FST} $
    
% \end{frame}

\section{Arithmetic Functions and Predicates}

\begin{frame}{PRED}
    
    We now have the data structure required to implement the algorithm for calculating the predecessor of a Church numeral. 
    
    \vspace{0.5cm}
    
    First we write a function which takes in a pair $p = (a,b)$ of Church numerals and outputs the pair consisting of the successor of the first (SUCC $a$) in the pair, together with the first $a$ in the pair.
    
    $$ \Psi = \lambda p. \ \text{PAIR} \ (\text{SUCC} \ (p \ \text{FST})) \ (p \ \text{FST}) $$
    
    Now we need to iterate this $n$ times on the input pair ZERO ZERO and retrieve the second element.
    
    $$\text{PRED} = \lambda n. \ (n \ \Psi \ (\text{PAIR ZERO ZERO})) \ \text{SCD}$$
    
    
\end{frame}

\begin{frame}{Example}
    
    PRED ONE
    \vspace{6cm}
     
\end{frame}

\begin{frame}{SUB}

    Given Church numerals $m,n$ how do we construct the Church numeral representing $m-n$?
    
    \vspace{0.3cm}
    
    $\text{SUB} = \lambda m. \ \lambda n. \ \lambda u. \ \lambda v. \ $
    
    \vspace{6cm}
    
\end{frame}

\begin{frame}{Example}
    
    SUB TWO ONE
    \vspace{6cm}
         
\end{frame}

\begin{frame}{GREATER?}

    Given Church numerals $m,n$ how do we test if one is larger than the other?
    
    \vspace{0.3cm}
    
    $\text{GREATER?} = \lambda m. \ \lambda n. \ $
    
    \vspace{6cm}  

\end{frame}

\begin{frame}{Example}

    GREATER? ONE ONE
    \vspace{6cm}
    
\end{frame}

\begin{frame}{EQUAL?}

    Given Church numerals $m,n$ how do we test if they are equal?
    
    \vspace{0.3cm}

    $\text{EQUAL?} = \lambda m. \ \lambda n. \ $

\vspace{6cm}      
\end{frame}

\begin{frame}{Example}

    EQUAL? ONE ZERO
    \vspace{6cm}
    
\end{frame}

\section{Summary}

\begin{frame}{Summary}

{\bf Arithmetic Functions}
\begin{align*}
    \text{SUCC} &= \lambda n. \ \lambda u. \ \lambda v. \ u(nuv) \\
    \text{SUM} &= \lambda m. \ \lambda n. \ \lambda u. \ \lambda v. \ mu(nuv)\\
    \text{MULT} &= \lambda m. \ \lambda n. \ \lambda u. \ \lambda v. \ m(nu)v \\
    \text{EXP} &= \lambda e. \ \lambda b. eb \ \\
    \text{PRED} &= \lambda n.  \ (n \ \Psi \ (\text{PAIR ZERO ZERO})) \ \text{SCD}  \\
    \text{SUB} &= \lambda m. \ \lambda n. \ \lambda u. \ \lambda v. \ (n \ \text{PRED} \ m) \ u \ v
\end{align*}

{\bf Arithmetic Predicates}
\begin{align*}
    \text{ZERO?} &= \lambda m. \ m \ (\lambda x. \ \text{FALSE}) \ \text{TRUE} \ \\
    \text{GREATER?} &= \ \lambda m. \ \lambda n. \ \text{ZERO?} \ (\text{SUB} \ n \ m) \\
    \text{LESS?} &= \ \lambda m. \ \lambda n. \ \text{ZERO?} \ (\text{SUB} \ m \ n) \\
    \text{EQUAL?} &= \lambda m. \ \lambda n. \ \text{AND} \ (\text{GREATER?} \ n \ m) \ (\text{LESS?} \ n \ m) 
\end{align*}

    
\end{frame}

\begin{frame}
	\frametitle{Further Reading}
	
	Here are some recommended reading to follow up on the lecture content. Some are freely available online. 

	\vspace{0.4cm}

	Type Theory and Functional Programming, \emph{Simon Thompson}
	
	\vspace{0.1cm}
	
	\begin{itemize}
		\item Stanford Encyclopedia Articles: 
			\begin{enumerate}
				\item The Lambda Calculus
				\item Type Theory
				\item Church's Type Theory
			\end{enumerate}
	\end{itemize}

	Church's original papers are worth visiting, although more work than Turing's paper. 

	An Unsolvable Problem of Elementary Number Theory, \emph{Alonso Church}

	\vspace{0.2cm}

	A Note on the Entscheidungsproblem, \emph{Alonso Church}
	
\end{frame}





\end{document}
