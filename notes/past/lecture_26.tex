% Load required themes and packages.
\documentclass{beamer}
\usetheme{Pittsburgh}
\usecolortheme{default}
\useinnertheme{default}
\useoutertheme{default}
\usefonttheme{structurebold}
% Math related packges.
\usepackage{amsmath}
\usepackage{cancel}
\usepackage{soul}

% Natural Deduction package
\usepackage{proof}

% Package for hyperlinks
\usepackage{hyperref}

% Remove navigation bar
\beamertemplatenavigationsymbolsempty

% Bibliography
% \usepackage[style=verbose]{biblatex}
% \bibliography{deltaBib.bib}
% Command line sequence:
% pdflatex *filename*
% biber *filename*
% pdflatex *filename*

\title{Lecture 26: Normal Forms and Reduction}
\author{MATH230}
\institute{Te Kura P\=angarau $\vert$ School of Mathematics and Statistics \\ Te Whare W\=ananga o Waitaha $\vert$ University of Canterbury}
\date{}

% Document body starts here.
\begin{document}


% Title frame
\begin{frame}

  \titlepage

\end{frame}

% Table of contents page
\begin{frame}
  \frametitle{Outline}

  \tableofcontents

\end{frame}

\section{Examples}

\begin{frame}
	\frametitle{Example}

	Perform a step of $\beta$-reduction on the following $\beta$-redex

	$$(\lambda x. \ x \ (\lambda y. \ x \ y)) \ z$$

	\vspace{7cm}	
	
\end{frame}

\begin{frame}
	\frametitle{Example}

	Use $\alpha$-reduction to relabel this $\lambda$-expression so that there are no clashes between bound and free variables. 

	$$(\lambda x. \ x \ (\lambda y. \ x \ y)) \ y$$

	\vspace{2cm}

	Now reduce the expression using $\beta$-reduction.

	\vspace{3cm}
	
	
\end{frame}

\section{$\beta$-reduction Strategies}

\begin{frame}
	\frametitle{Reduction Strategies}

	One $\lambda$-expression can consist of multiple $\beta$-redex. If this is the case, then different strategies may give different outcomes. 

	$$(\lambda y. \ \lambda z. \ z) \ ((\lambda x. \ x \ x) \ (\lambda x. \ x \ x))$$

	\vspace{3cm}

	If we are to define an automatic model for computation, then we can't have any ambiguity in the process. One needs to decide ahead of time how to deal with these choices. 
	
	\vspace{0.5cm}
	
	Two primary strategies are known as call-by-name and call-by-value.

\end{frame}

\begin{frame}
	\frametitle{Call-by-name}

	Once the leftmost $\beta$-redex is identified, this strategy calls the leftmost abstraction leaving the input (potentially) unevaluated.

	$$(\lambda y. \ \lambda z. \ z) \ ((\lambda x. \ x \ x) \ (\lambda x. \ x \ x))$$

	\vspace{4.5cm}
	
	This strategy is also known as either \emph{normal order} or \emph{lazy evaluation}.

\end{frame}

\begin{frame}
	\frametitle{Call-by-value}

	Once the leftmost $\beta$-redex is identified, this strategy calls the innermost abstraction. Calls the value of the outermost $\beta$-redex. 

	$$(\lambda y. \ \lambda z. \ z) \ ((\lambda x. \ x \ x) \ (\lambda x. \ x \ x))$$

	\vspace{4.5cm}
	
	This strategy is also known as either \emph{strict evaluation} or \emph{eager evaluation}.

\end{frame}

\begin{frame}
	\frametitle{Normal Form}

	We say a $\lambda$-expression is in normal form if it does not contain any $\beta$-redex. If a $\lambda$-expression is $\beta$-equivalent to a $\lambda$-expression in normal form, then it is unique upto $\alpha$-reduction. 

	\vspace{0.5cm}

	If a $\lambda$-expression has a normal form, then call-by-name evaluation will result in that normal form. 

	\vspace{0.5cm}

	{\bf Observation:} As a consequence of the above, if you can show that call-by-name evaluation does not terminate, then you may conclude that the $\lambda$-expression does not have a normal form. 

	\vspace{3cm}

\end{frame}

\begin{frame}
	\frametitle{Example: Call-by-Value}

	Perform a step of $\beta$-reduction on the following $\beta$-redex

	$$(\lambda y. \ y \ a) \ ((\lambda x. \ x) \ (\lambda z. \ (\lambda u. \ u) \ z))$$

	\vspace{7cm}	
		
\end{frame}

\begin{frame}
	\frametitle{Example: Call-by-Name}

	Perform a step of $\beta$-reduction on the following $\beta$-redex

	$$(\lambda y. \ y \ a) \ ((\lambda x. \ x) \ (\lambda z. \ (\lambda u. \ u) \ z))$$

	\vspace{7cm}	
		
\end{frame}

\begin{frame}
	\frametitle{Reduction Graphs}

	You may come across graphs of $\lambda$-expressions for which the different paths through the graph represent different evaluation strategies. 

	$$(\lambda x. \ 3\cdot x) \ ((\lambda x. \ x + 1) \ 2) $$
	% Here I am assuming there are lambda expressions that can represent natural numbers and the arithmetic thereof. We will come to that later. 
	% However it's nice to point out that we can substitute a lambda expression for a name as we only need to know the lambda expression for IF we ever call it. 
	\vspace{5cm}

\end{frame}

\section{Encoding Booleans}
\begin{frame}
	\frametitle{Common Expressions}

	Some expressions that are common enough, or illustrative of some point, have been granted names. As we saw in the previous example, when writing $\lambda$-expressions it can be enough to know the name - until we need to evaluate that expression on some input. 

	\vspace{0.5cm}

	\begin{itemize}
		\item[] $I = \lambda x. \ x$
		\item[] $\omega = \lambda x. \ x \ x$
		\item[] $\Omega = \omega \omega$
		\item[] $\text{Apply} = \lambda f. \ \lambda a. \ (f \ a)$
	\end{itemize}

	\vspace{0.5cm}

	We will now talk about how we can encode ideas from logic and arithmetic into the $\lambda$-calculus. This requires giving $\lambda$-expression encoding for Booleans (True/False), natural numbers, and arithmetic functions.

\end{frame}

\begin{frame}
	\frametitle{Conditional Execution}

	Remember: everything (!) is a $\lambda$-expression. 

	\vspace{0.5cm}

	If we want to implement logic in the $\lambda$-calculus, then we need $\lambda$-expressions that \emph{behave like} True and False, propositions, propositional connectives, quantifiers etc.

	\vspace{0.5cm}

	How do True and False \emph{behave}? One use is conditional branching.

	$$\text{COND} = \lambda f. \ \lambda g. \ \lambda c. \ ((c \ f) \ g) $$

	\vspace{3cm}

\end{frame}

\begin{frame}
	\frametitle{TRUE}

	COND is an expression with three arguments: $a,b,c$. Which should be equivalent to the first argument $a$ if $c = \text{TRUE}$ and the second argument $b$ if $c = \text{FALSE}$.

	$$\text{COND} \ a \ b =_{\beta} \lambda c. \ ((c \ a) \ b) $$

	If we just put in the first two arguments, we get a function waiting for a Boolean to tell it what to do. 

	\vspace{0.5cm}

	Our $\lambda$-expression for TRUE needs to ignore the second input $b$ 

	$$\text{TRUE} = \lambda x. \ \lambda y. \ x$$


	
\end{frame}

\begin{frame}
	\frametitle{TRUE}

	Perform $\beta$-reduction on the following $\lambda$-expression until the $\lambda$-expression is in normal form. 

	$$ (((\text{COND} \ a) \ b) \ \text{TRUE}) = ((((\lambda f. \ \lambda g. \ \lambda c. \ ((c \ f) \ g) ) \ a) \ b) \ (\lambda x. \ \lambda y. \ x))$$

	\vspace{5cm}

\end{frame}


\begin{frame}
	\frametitle{FALSE}
	
	If the $\lambda$-expression for TRUE ignores the second function, then $\lambda$-expression for FALSE needs to ignore the first argument.

	$$ \text{FALSE} =\hspace{3cm} $$

	\vspace{5cm}
	
\end{frame}

\begin{frame}
	\frametitle{FALSE}

	Perform $\beta$-reduction on the following $\lambda$-expression until the $\lambda$-expression is in normal form. 

	$$ ((\text{COND} \ a) \ b)(\text{FALSE}) =_{\beta} \lambda c. \ ((c \ a) \ b)(\lambda x. \ \lambda y. \ y)$$

	\vspace{5cm}

\end{frame}

\begin{frame}
	\frametitle{Propositional Connectives}

	TRUE and FALSE should behave appropriately with some implementation in $\lambda$-expressions of the propositional connectives 
	
	\begin{itemize}
		\item[] NOT 
		\item[] AND 
		\item[] OR 
		\item[] IMP(LICATION)
		\item[] NAND 
		% Equiv BOOLEAN is another interesting one. 
	\end{itemize}

	The trick is to think about these in terms of selecting the first or second argument. 
	
	\vspace{0.3cm}

	{\bf Example:} If the first input to AND is TRUE, then which input should the AND expression return? 

	\vspace{0.3cm}
	
	For some of these, it may just be easier to build them once you have NOT, OR, and AND.

\end{frame}

\begin{frame}
	\frametitle{NOT}

	\begin{center}
		$\begin{array}{ l | c }			
			A &  \text{NOT} A\\
			\cline{1 - 2}
			T &  F \\ 
			F &  T
		\end{array}$
	\end{center}

	$\text{TRUE} = \lambda x. \ \lambda y. \ x$ \hspace{4cm} $\text{FALSE} = \lambda x. \ \lambda y. \ y$

	\vspace{5cm}

\end{frame}

\begin{frame}
	\frametitle{EXAMPLE}

	Perform $\beta$-reduction on the following $\lambda$-expression until the $\lambda$-expression is in normal form.
	
	\vspace{0.5cm}

	$\text{NOT} \ \text{FALSE}$

	\vspace{5cm}

\end{frame}

\begin{frame}
	\frametitle{Further Reading}
	
	Here are some recommended reading to follow up on the lecture content. Some are freely available online. 

	\vspace{0.4cm}

	Type Theory and Functional Programming, \emph{Simon Thompson}
	
	\vspace{0.1cm}
	
	\begin{itemize}
		\item Stanford Encyclopedia Articles: 
			\begin{enumerate}
				\item The Lambda Calculus
				\item Type Theory
				\item Church's Type Theory
			\end{enumerate}
	\end{itemize}

	Church's original papers are worth visiting, although more work than Turing's paper. 

	An Unsolvable Problem of Elementary Number Theory, \emph{Alonso Church}

	\vspace{0.2cm}

	A Note on the Entscheidungsproblem, \emph{Alonso Church}
	
\end{frame}





\end{document}
