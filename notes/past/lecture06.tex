% Load required themes and packages.
\documentclass{beamer}
\usetheme{Pittsburgh}
\useinnertheme[shadow]{rounded}
\setbeamercolor{block title}{bg=blue!30, fg=black}
\setbeamercolor{block body}{bg=blue!20}
\usecolortheme{default}
\useinnertheme{default}
\useoutertheme{default}
\usefonttheme{structurebold}
% Math related packges.
\usepackage{amsmath}
\usepackage{cancel}

% Natural Deduction package
\usepackage{proof}

% Package for hyperlinks
\usepackage{hyperref}

% Remove navigation bar
\beamertemplatenavigationsymbolsempty

% Bibliography
% \usepackage[style=verbose]{biblatex}
% \bibliography{deltaBib.bib}
% Command line sequence:
% pdflatex *filename*
% biber *filename*
% pdflatex *filename*

\newtheoremstyle{indentDefn}
{\topsep} % Space above
{\topsep} % Space below
{\it} % Body font
{2cm} % Indent amount
{\bf} % Theorem head font
{:} % Punctuation after theorem head
{0.5em} % Space after theorem head
{} % Theorem head spec

\theoremstyle{indentDefn} \newtheorem{defn}[]{Definition}

\title{Lecture 6: Classical Logic}
\author{MATH230}
\institute{Te Kura P\=angarau $\vert$ School of Mathematics and Statistics \\ Te Whare W\=ananga o Waitaha $\vert$ University of Canterbury}
\date{}

% Document body starts here.
\begin{document}


% Title frame
\begin{frame}

  \titlepage

\end{frame}

% Table of contents page
\begin{frame}
  \frametitle{Outline}

  \tableofcontents

\end{frame}

\section{Examples}

\begin{frame}
  \frametitle{Example: Minimal Logic}

	Show $\{ A \lor B, \ (A \lor C)\rightarrow D, \ B \rightarrow D  \} \ \vdash D$
	
	\vspace{7cm}



\end{frame}

\section{Intuitionistic Logic}

\begin{frame}
	\frametitle{Ex Falso Sequitur Quadlibet}
	
	So far, we have not made much mention of how to deal with the derivation of $\bot$ absurdity. 
	
	\vspace{0.5cm}
		
	If $^{\Sigma}_{\bot}\mathcal{D}$ is a deduction of $\bot$ from $\Sigma$, then
	
	\begin{center}		
		$\begin{array}{c}		
		\infer[\bot]{\alpha}
		{\begin{array}{c} \Sigma \\ \mathcal{D} \\ \bot \end{array}}
		\end{array}$
	\end{center}

	is a derivation of $\alpha$ from the assumptions $\Sigma$.
	
	\vspace{0.5cm}
	
	\emph{Anything you want follows from a falsehood.} Recall this is valid. 
	 
\end{frame}

\begin{frame}
	\frametitle{Disjunctive Syllogism}
	
	Show $\{\alpha \lor \beta, \lnot\beta\} \vdash \alpha$
	\vspace{7cm}
	
\end{frame}

\begin{frame}
	\frametitle{Double Negation Elimination}
	
	Show $\lnot\lnot \alpha \vdash \alpha$ 
	\vspace{5cm}
	
	\pause
	% LaTeX the derivation here. 
	
	{\bf Ex falso does not give us a proof.} In fact we have shown the following: $\{\lnot\lnot\alpha, \lnot\alpha\} \vdash \alpha$.
	
\end{frame}

\section{Classical Logic}


\begin{frame}
	\frametitle{Reductio Ad Absurdum}
	
	If $^{\Sigma}_{\bot}\mathcal{D}$ is a deduction of $\bot$ from $\Sigma$, then
	
	\begin{center}		
		$\begin{array}{c}		
		\infer[RAA]{\alpha}
		{\begin{array}{c} \hline \cancel{\lnot\alpha} \\ \Sigma \\ \mathcal{D} \\ \bot \end{array}}
		\end{array}$
	\end{center}
	
	is a derivation of $\alpha$ from the assumptions $\Sigma \backslash\{\lnot\alpha\}$.
	
	\vspace{0.5cm}
	
	If absurdity follows from $\lnot\alpha$, then we may conclude $\alpha$ {\bf and discharge $\lnot \alpha$ from our assumptions.}
	 
\end{frame}

\begin{frame}
	\frametitle{Double Negation}	
	
	Show $\lnot\lnot\alpha \vdash \alpha$
	\vspace{7cm}	
	
	% Notice the difference if Ex Falso is used - then we prove something very different!
\end{frame}

\begin{frame}
	\frametitle{Law of Excluded Middle}
	
	Show $\vdash \alpha \lor \lnot \alpha$
	\vspace{7cm}
	
	% Notice the difference if Ex Falso is used - then we prove something very different!	
\end{frame}

\begin{frame}
	\frametitle{Example}
	Show $\{\alpha \rightarrow \beta, \hspace{0.2cm} \alpha \rightarrow \lnot\beta\} \vdash \lnot\alpha$
	\vspace{7cm}
	
	
	% The last step can be achieved by RAA... but it can also be seen as implication introdcution. It's minimal!
	
\end{frame}

\section{Derived Rules of Inference}

\begin{frame}
	\frametitle{Derived Rules of Inference}
	
	Proofs can be simplified by using results already proved. You may, in the course of a proof, use any result that has been proven in class or previously in a tutorial. However, when substituting previous proofs, you must bring all of the premises with the conclusion. 
	
	\vspace{0.5cm}
	
	We have already seen this with the use of \emph{modus tollens} (MT) in some examples.
	
	\vspace{0.5cm} 
	
	This can help keep proofs manageable and neat.

\end{frame}

\begin{frame}
	\frametitle{Example: Substituting LEM}
	
	%Earlier we showed $\vdash \alpha \lor \lnot\alpha$ so we can call on this theorem when ever it may help us.
	Show $\alpha \rightarrow \beta \vdash \lnot\alpha \lor \beta$
	
	\vspace{7cm}
		
	
\end{frame}





\begin{frame}
  \frametitle{Further Reading}


Below is a selection of resources that I used to prepare the lecture. You might like to read over them yourself to help get a more complete picture of the topics discussed. 

\vspace{0.5cm}

\begin{itemize}
	\item L$\exists \forall$N, \emph{Logic and Proof}. Section: 5  
	\item Logic Matters, \emph{An Introduction to Formal Logic}: Sections 17, 23, and 24.8
\end{itemize}


\end{frame}





\end{document}
