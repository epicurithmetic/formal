% Load required themes and packages.
\documentclass{beamer}
\usetheme{Pittsburgh}
\useinnertheme[shadow]{rounded}
\setbeamercolor{block title}{bg=blue!30, fg=black}
\setbeamercolor{block body}{bg=blue!20}
\usecolortheme{default}
\useinnertheme{default}
\useoutertheme{default}
\usefonttheme{structurebold}
% Math related packges.
\usepackage{amsmath}
\usepackage{cancel}

% Natural Deduction package
\usepackage{proof}

% Package for hyperlinks
\usepackage{hyperref}

% Remove navigation bar
\beamertemplatenavigationsymbolsempty

% Bibliography
% \usepackage[style=verbose]{biblatex}
% \bibliography{deltaBib.bib}
% Command line sequence:
% pdflatex *filename*
% biber *filename*
% pdflatex *filename*

\newtheoremstyle{indentDefn}
{\topsep} % Space above
{\topsep} % Space below
{\it} % Body font
{2cm} % Indent amount
{\bf} % Theorem head font
{:} % Punctuation after theorem head
{0.5em} % Space after theorem head
{} % Theorem head spec

\theoremstyle{indentDefn} \newtheorem{defn}[]{Definition}

\title{Lecture 12: Proof Theory I}
\subtitle{Syntax of First Order Logic}
\author{MATH230}
\institute{Te Kura P\=angarau $\vert$ School of Mathematics and Statistics \\ Te Whare W\=ananga o Waitaha $\vert$ University of Canterbury}
\date{}

% Document body starts here.
\begin{document}


% Title frame
\begin{frame}

  \titlepage

\end{frame}

% Table of contents page
\begin{frame}
  \frametitle{Outline}

  \tableofcontents

\end{frame}

\section{Introduction}

\begin{frame}
  \frametitle{Syntax v. Semantics}

	Given a first order language $\mathcal{L}$ we have determined how to assign meaning to well-formed formulae. As the quantifers and predicates increase the scope of the language, the truth and falsity of formulae in $\mathcal{L}$ are difficult to determine. 
	
	\vspace{0.5cm} 
	
	As in the case of propositional logic, we have another method for analysing whether some formulae follow from others; the syntactic, proof theoretic, point of view. 
	
	\vspace{0.5cm} 
	
	We will now extend the natural deduction calculus to include rules of inference for the quantifiers. 
\end{frame}

\begin{frame}
	\frametitle{Universal Quantification}
	
	Suppose we had, in the course of a proof, deduced the statement $\forall x \alpha$, where $\alpha$ is a well-formed formula that (may) contain the variable $x$. What can we conclude from that? 
	
	\vspace{0.5cm}
	
	\begin{center}
	$\begin{array}{c}
		\infer[\forall E]{\hspace{0.3cm} ? \hspace{0.3cm}}{\begin{array}{c} 
			\Sigma \\
			\mathcal{D} \\ 
			\forall x \alpha			
			\end{array}}
	\end{array}$
	\end{center}

	Following our model theory interpretation of $\forall$ we should then be able to substitute any term, $t$, in place of each instance of the variable $x$ in $\alpha$.
\end{frame}

\section{Substitution, Free, and Bound Variables}

\begin{frame}
	\frametitle{Substitution Rules}
	
	If we don't take care when subsitituting for variables, then we can go astray. 
	
	\vspace{0.5cm}
	
	\begin{defn} If a variable, $x$, is used to quantify over $\alpha$ --- as in $\gamma = \forall x \alpha$ --- then we say any instance of $x$ in $\alpha$ is \emph{bound} in $\gamma$ by the quantifer. Similarly for the existential quantifier.  \end{defn}
	% Lecture: make clear that it's bound in the wff that includes the quantifier.
	
	\vspace{0.5cm} 
		
	\begin{defn} If a variable, $x$, is not bound in $\alpha$, then we say that it is free in $\alpha$.\end{defn}
		
	\vspace{0.5cm} 
	
	Bound and free variables act differently when interpreted; they play different roles. 
	
	% Lecture: Technically a variable can be both in the same formula. But, in that case, it's better to rename the bound variable. So you should never have a variable that is both.

	
\end{frame}

\begin{frame}
	\frametitle{Free vs. Bound Variables}
	
	We can translate the first-order expression: $\forall x (x < y)$ as ``every $x$ is less than \_\_\_\_". The interpretation of this statement depends on the interpretation of $y$; because $y$ is free. It's truth will depend on the interpretation of $y$.
	
	\vspace{0.5cm} 
	
	On the other hand the interpretation of $\exists y (\forall x (x < y))$ doesn't depend on $y$. As, in a model, there either exists something that satisfies this sentence, or there doesn't. 
	
	\vspace{0.5cm}

	{\bf For this reason we should not make free variables bound, or vice versa, when substituting.}
	
	
\end{frame}

\begin{frame}
	\frametitle{Examples}

	

	\begin{itemize}
		\item[]	$x_{2}$ is free for $x_{0}$ in $\exists x_{3} (P(x_{0},x_{3}))$, \vspace{0.2cm}
		\item[] $f(x_{0},x_{1})$ is not free for $x_{0}$ in $\exists x_{1} (P(x_{0},x_{3}))$, \vspace{0.2cm}
		\item[] $x_{5}$ is free for $x_{1}$ in $P(x_{1},x_{3})\rightarrow \exists x_{1} (Q(x_{1},x_{2}))$.
	\end{itemize}	

	
\end{frame}

\begin{frame}
	\frametitle{What Can Go Wrong?}
	
	Consider the formula 
	
	$$\forall x \exists y (y > x).$$
	
	In a model with domain the natural numbers, this asserts that there is always a larger natural number. 
	
	\vspace{0.5cm}
	
	Notice that $x$ is free in the formula $\exists y (y > x)$. So if we aren't careful and substitute $x \mapsto y+1$, then we obtain the statement 
	
	$$ \exists y (y > y+1)$$
	
	Which is not true in this interpretation. {\bf This problem arose because $y$ is bound in the formula, so we can't substitute $t=y+1$ for $x$.}
	
	\vspace{0.5cm}
	
	\footnotesize{Example from Section 7.3 L$\exists\forall$N}
	
\end{frame}

\section{Rules of Inference: Universal}

\begin{frame}
	\frametitle{Universal Elimination}
	
	If $^{\Sigma}_{\forall x \alpha}\mathcal{D}$ is a derivation, where $y$ is free for $x$ in $\alpha$, then
	
	\vspace{0.5cm}
	
	\begin{center}
		$\begin{array}{c}
		\infer[\forall E]{\alpha[x/y]}{\begin{array}{c} 
			\Sigma \\
			\mathcal{D} \\ 
			\forall x \alpha			
			\end{array}}
		\end{array}$
	\end{center}
	
	is a derivation of $\alpha[x/y]$ from hypotheses $\Sigma$. 
	
\end{frame}

\begin{frame}
	\frametitle{Example}
	
	Every human is mortal. Sherlock is Human. Therefore, Sherlock is mortal. 
	
	\vspace{6cm}
	
\end{frame}

\begin{frame}
	\frametitle{Proof of For All}
	
	What does it mean to prove something ``for all x"?
	
	\vspace{0.5cm} 
	
	A proof of the irrationality of $\sqrt{2}$ starts as follows: 
	
	\vspace{0.2cm}
	
	\begin{center}
		Assume there are some integers $A,B$ such that $\sqrt{2} = A/B$.... Eventually deriving a contradiction from this assumption.
	\end{center}
	
	The proof itself, in the end, assumes \emph{nothing} of the integers $A,B$ apart from their (possible) existence. Thus it shows that no matter which integers are in place of $x,y$, the formula 
	
	$$(x^{2} = 2y^{2})$$
	
	can have no solutions and hence we may conclude 
	
	$$\forall x,y \ \lnot (x^{2} = 2y^{2})$$
	
\end{frame}

\begin{frame}
	\frametitle{Universal Introduction}
	
	If $^{\Sigma}_{\alpha[x/y]}\mathcal{D}$ is a derivation, where $y$ does not appear free in $\Sigma$ nor $\alpha$, then
	
	\vspace{0.5cm}
	
	\begin{center}
		$\begin{array}{c}
		\infer[\forall I]{\forall x \alpha}{\begin{array}{c} 
			\Sigma \\
			\mathcal{D} \\ 
			\alpha[x/y]
			\end{array}}
		\end{array}$
	\end{center}
	
	is a derivation of $\forall x \alpha$ from hypotheses $\Sigma$. 
	
	\vspace{0.5cm} 
	
	{\bf Properties of $y$ can't play a role in the proof of $\alpha[x/y]$. They can't appear in the hypotheses $\Sigma$ of the proof.} \\
	
	{\bf Nor can any y be free in alpha. You should be replacing all instances of y with an x when using the $\forall$ introduction.}
	
\end{frame}

\begin{frame}
	\frametitle{Example}
	
	Show $\{\forall x (Px \rightarrow Qx), \hspace{0.5em} \forall x Px\} \hspace{0.5em} \vdash \hspace{0.5em} \forall x Qx$
	
	\vspace{7cm}
	
\end{frame}

\section{Examples}

\begin{frame}
	\frametitle{Example}
	
	Show $\{\forall x (Px \rightarrow Qx), \hspace{0.5em} \forall x (Qx \rightarrow Rx)\} \hspace{0.5em} \vdash \hspace{0.5em} \forall x (Px \rightarrow Rx)$
	
	% Make a point about the $a$ being part of the assumptions, but by the time the \forall comes in we have already discharged it from the assumption list. 
	
	\vspace{7cm}
	
\end{frame}

\begin{frame}
	\frametitle{Example: When Things Go Wrong}
	
	Sherlock is a person. Gladstone is not a person. Therefore, Moriarty is the Queen.
	
	\vspace{0.5cm}
	
	Show $\{Ps, \ \lnot Pg\} \hspace{0.5em} \vdash \hspace{0.5em} Q$
	
	% Make a point about the $a$ being part of the assumptions. This causes the problem, in the first step.
	
	\begin{center}
		$\begin{array}{c}
			\infer[\bot]{Q}
				{\infer[MP]{ \ \bot \ }{\lnot Pg & \infer[\forall E]{ \ Pg \ }{\infer[\forall I]{ \ \forall x Px \ }{Ps\vspace{0.2cm}}}}}
		\end{array}$
	\end{center}
	
	\pause
	{\bf From one instance (Sherlock's Personhood) we cannot assume all objects are people.}
	
\end{frame}

\begin{frame}
  \frametitle{Further Reading}


Below is a selection of resources that I used to prepare the lecture. You might like to read over them yourself to help get a more complete picture of the topics discussed. 

\vspace{0.5cm}

\begin{itemize}
	\item L$\exists \forall$N, \emph{Logic and Proof}. Section: 8.  
	\item Logic Matters, \emph{An Introduction to Formal Logic}: 32 and 33.
\end{itemize}

\end{frame}





\end{document}
