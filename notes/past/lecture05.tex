% Load required themes and packages.
\documentclass{beamer}
\usetheme{Pittsburgh}
\useinnertheme[shadow]{rounded}
\setbeamercolor{block title}{bg=blue!30, fg=black}
\setbeamercolor{block body}{bg=blue!20}
\usecolortheme{default}
\useinnertheme{default}
\useoutertheme{default}
\usefonttheme{structurebold}
% Math related packges.
\usepackage{amsmath}
\usepackage{cancel}

% Natural Deduction package
\usepackage{proof}

\usepackage{hyperref}

% Remove navigation bar
\beamertemplatenavigationsymbolsempty

% Bibliography
% \usepackage[style=verbose]{biblatex}
% \bibliography{deltaBib.bib}
% Command line sequence:
% pdflatex *filename*
% biber *filename*
% pdflatex *filename*

\newtheoremstyle{indentDefn}
{\topsep} % Space above
{\topsep} % Space below
{\it} % Body font
{2cm} % Indent amount
{\bf} % Theorem head font
{:} % Punctuation after theorem head
{0.5em} % Space after theorem head
{} % Theorem head spec

\theoremstyle{indentDefn} \newtheorem{defn}[]{Definition}

\title{Lecture 5: Rules of Inference}
\author{MATH230}
\institute{Te Kura P\=angarau $\vert$ School of Mathematics and Statistics \\ Te Whare W\=ananga o Waitaha $\vert$ University of Canterbury}
\date{}

% Document body starts here.
\begin{document}


% Title frame
\begin{frame}

  \titlepage

\end{frame}

% Table of contents page
\begin{frame}
  \frametitle{Outline}

  \tableofcontents

\end{frame}

\section{Rules of Inference}

\begin{frame}
	\frametitle{Hypothetical Reasoning}
	
	In mathematics we often prove statements of the following hypothetical form: ``If ..., then ..." 
	
	\vspace{0.5cm}
	
	{\bf Example:} Suppose $f$ is a smooth function. If $f$ is continuous, then it is differentiable.
	
	\vspace{0.5cm}
	
	The proof of such a statement will assume the hypothesis of continuity and show that it implies differentiability.
	
	\vspace{0.5cm} 
	
	The conclusion is the entire implication, not just differentiablity. 
	
	\vspace{3cm}
	
\end{frame}

\begin{frame}
  \frametitle{$\rightarrow$ Introduction}

	If $^{\Sigma}_{\beta}\mathcal{D}$ is a deduction of $\beta$ from $\Sigma$, then

	\begin{center}		
		$\begin{array}{c}		
			\infer[\rightarrow I]{\alpha \rightarrow \beta}
				{\begin{array}{c} \hline \cancel{\alpha} \\ \Sigma \\ \mathcal{D} \\ \beta \end{array}}
		\end{array}$
	\end{center}

	is a deduction of $\alpha \rightarrow \beta$ from hypotheses $\Sigma \backslash \{\alpha\}$.
	
	\vspace{1.5cm} 
	
	{\bf Note: }As the assumption $\alpha$ is struck out after this deduction, we are free to use $\alpha$ \emph{even if it is not in} $\Sigma$ when using implication introduction. 

\end{frame}

\begin{frame}
	\frametitle{Example}
	
	Show $\{\alpha \rightarrow \beta, \ \beta \rightarrow \gamma\} \ \vdash \ \alpha \rightarrow \gamma$
	
	\vspace{7cm}
	
\end{frame}

\begin{frame}
	\frametitle{Falsum}
	
	We introduce the logical constant $\bot$ (falsum or absurdity) to define the syntactic form of the $\lnot$ connective. We make the following definition: 
	
	$$\lnot \alpha:= \alpha \rightarrow \bot$$
	
	\vspace{0.5cm}
	
	\begin{center}
		$\begin{array} { c c c }
		
		\infer[MP]{\bot}{\alpha \quad & \alpha \rightarrow \bot}
		
		& \hspace{3cm} &
		
		\infer[\rightarrow I]{\alpha \rightarrow \bot}{\begin{array}{c} 
			\hline \cancel{\alpha} \\
			\mathcal{D} \\ 
			\bot			
		\end{array}}		
		
		\end{array}$
	\end{center}
	
	\vspace{1cm}
	
	Falsum $\bot$ is to be thought of as ``absurdity" or ``contradiction". 
	
\end{frame}

\begin{frame}
	\frametitle{Example: Modus Tollens}	
	
	Show $\{\alpha \rightarrow \beta, \quad \lnot\beta\} \vdash \lnot \alpha$
	
	\vspace{7cm}
	
\end{frame}

\begin{frame}
	\frametitle{Contradiction Implies Absurdity}
	
	Show $A \land \lnot A \ \vdash \ \bot$
	
	\vspace{7cm}
	
\end{frame}

\begin{frame}
	\frametitle{$\land$ Introduction}
	
	If $^{\Sigma_{1}}_{\alpha}\mathcal{D}_{1}$ and $^{\Sigma_{2}}_{\beta}\mathcal{D}_{2}$ are deductions, then

	\vspace{0.5cm}

	\begin{center}
		$\begin{array}{c}		
			\infer[\land I]{\alpha \land \beta}{\begin{array}{c} \Sigma_{1} \\ \mathcal{D}_{1} \\ \alpha \end{array} \quad & \begin{array}{c} \Sigma_{2} \\ \mathcal{D}_{2} \\ \beta \end{array}}	
		\end{array}$
	\end{center}

	is a deduction of $\alpha \land \beta$ from $\Sigma_{1} \cup \Sigma_{2}$. 
\end{frame}

\begin{frame}
	\frametitle{$\land$ Elimination}
	
	If $^{\Sigma}_{\alpha \land \beta}\mathcal{D}$ is a deduction of $\alpha \land \beta$ from $\Sigma$, then
	
	\begin{center}	
		
		$\begin{array}{ c c }		
		\begin{array}{c}		
			\infer[\land E_{L}]{\alpha}
			{\begin{array}{c} \Sigma \\ \mathcal{D} \\ \alpha \land \beta \end{array}}
		\end{array}
		
		\quad
		
		&
		
		\quad
		
		\begin{array}{c}		
			\infer[\land E_{R}]{\beta}
			{\begin{array}{c} \Sigma \\ \mathcal{D} \\ \alpha \land \beta \end{array}}
		\end{array}
		\end{array}$
	\end{center}
	
	are deductions of $\alpha$ and $\beta$ from $\Sigma$.
	
\end{frame}

\begin{frame}
	\frametitle{Example: Commutativity of $\land$}
	
	Show $\alpha \land \beta \dashv\vdash \beta \land \alpha$
		
	\vspace{7cm}
	
\end{frame}

\begin{frame}
	\frametitle{Example: Idempotence of $\land$}
	
	Show $\alpha \land \alpha \dashv\vdash \alpha$
	
	\vspace{7cm}
	
\end{frame}


\begin{frame}
	\frametitle{$\lor$ Introduction}
	
	If $^{\Sigma}_{\alpha}\mathcal{D}$ is a derivation of $\alpha$ from $\Sigma$, then 
	
	\begin{center}
		
	$\begin{array}{ c c c }	
		\begin{array}{ c c }		
			\begin{array}{c}		
				\infer[\lor I_{R}]{\alpha \lor \beta}
				{\begin{array}{c} \Sigma \\ \mathcal{D} \\ \alpha \end{array}}
			\end{array}
		\end{array}
		
		&
		
		&
		
		\begin{array}{ c c }		
			\begin{array}{c}		
				\infer[\lor I_{R}]{\beta \lor \alpha}
				{\begin{array}{c} \Sigma \\ \mathcal{D} \\ \alpha \end{array}}
			\end{array}
		\end{array}		
		
		
	\end{array}$
	\end{center}
	
	are derivations of $\alpha \lor \beta$ and $\beta \lor \alpha$ from $\Sigma$. 
	
	\vspace{1cm}
	
	{\bf Note:} We are free to choose $\beta$ as, if we know $\alpha$ to be the case, then $\alpha \lor \beta$  is necessarily the case \emph{for any }$\beta$. 
	
\end{frame}

\begin{frame}
	\frametitle{$\lor$ Elimination}
	
	If $^{\Sigma_{1}}_{\alpha\lor\beta}\mathcal{D}_{1}$, $^{\Sigma_{2}}_{\alpha\rightarrow\gamma}\mathcal{D}_{2}$, and $^{\Sigma_{2}}_{\beta\rightarrow\gamma}\mathcal{D}_{3}$ are derivations, then
	
	\begin{center}
	$\begin{array}{ c }
	
		\infer[\lor_{E}]{\gamma}
			{
			\begin{array}{c} \Sigma_{1}  \\ \mathcal{D}_{1} \\ \alpha\lor\beta \end{array}
			 & 
			\begin{array}{c} \Sigma_{2}  \\ \mathcal{D}_{2} \\ \alpha\rightarrow\gamma \end{array}				
			 & 
			\begin{array}{c} \Sigma_{2}  \\ \mathcal{D}_{2} \\ \beta\rightarrow\gamma \end{array}				 
			}	
	
	\end{array}$
	\end{center}
	
	is a derivation of $\gamma$ from $\Sigma_{1}\cup\Sigma_{2}\cup\Sigma_{3}$.
	
	\vspace{1cm}
	
	{\bf Note:} You can't remove one of the arguments from a disjunction. Knowledge of $\alpha \lor \beta$ is not sufficient to conclude either $\alpha$ or $\beta$ alone.
	
\end{frame}

\section{Minimal Logic}

\begin{frame}
	\frametitle{Minimal Logic}
	
	Together the rules of inference that we've given so far define \emph{minimal} logic. They include much, but not all, of the logical inferences that practising mathematicians might use in a proof. What they do include is uncontroversial.
	
	\vspace{0.5cm}
	
	However it is not universally agreed upon how minimal logic should be extended. There are philosophical differences among mathematicians and logicians about what other rules of inference should be included. 
	
	\begin{itemize}
		\item Classical logic 
		\item Intuitionistic logic
		\item Modal logic		
	\end{itemize}
	
	{\bf Report:} These philosophical differences and different logics would make for an interesting report topic. 
	
\end{frame}

\begin{frame}
  \frametitle{Further Reading}


Below is a selection of resources that I used to prepare the lecture. You might like to read over them yourself to help get a more complete picture of the topics discussed. 

\vspace{0.5cm}

\begin{itemize}
	\item L$\exists \forall$N, \emph{Logic and Proof}. Section: $3$ (and $4$ if you're interested)  
	\item Logic Matters, \emph{An Introduction to Formal Logic}. Sections: Interlude, $20 - 22$
\end{itemize}

\end{frame}





\end{document}
